Users' privacy requirements for their sensitive data on web applications are highly contextual and constantly in flux.  %
But today, achieving this flexible privacy remains out of reach, in part because
of the many challenges facing web developers in building services that can adapt
to changing privacy needs. Consequently, services often provide only
coarse-grained, blunt tools that result in all-or-nothing exposure of users’
private information.

This thesis identifies the need for new systems that help developers support
automated, fine-grained controls over user data that provide users with flexible
privacy, and grant them more nuanced controls over their data that don't exist
today.
%
To address the challenges associated with flexible privacy in web applications,
this thesis introduces Edna, a system that helps web developers realize
\emph{disguised data}, a new state of user data that bridges the gap between
permanently deleted data, and data left accessible and vulnerable to the
application.  With \sys, users can remove their data without permanently losing
their accounts, anonymize their old data, and selectively dissociate personal
data from public profiles.
%
