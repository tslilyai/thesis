% -*- TeX-master: "paper.tex"; TeX-PDF-mode: t -*-

\begingroup
%\fontfamily{phv}\large\selectfont

\newgeometry{left=1in,right=1in}

\begin{center}
Massachusetts Institute of Technology \\
Department of Electrical Engineering and Computer Science
\end{center}

\begin{center}
Proposal for Thesis Research in Partial Fulfillment \\
of the Requirements for the Degree of \\
Doctor of Philosophy
\end{center}

\vspace{1em}
\noindent Title: \mytitle

\vspace{1em}
\noindent\begin{tabular}{@{}ll}
Submitted by: & Lillian Tsai\\
              & Stata Center\\
              & Cambridge, MA 02139
\end{tabular}

\vspace{1em}
\noindent Signature of author: \rule{5cm}{0.1pt}

\vspace{1em}
\noindent Date of Submission: \mydate

\vspace{1em}
\noindent Expected Date of Completion: \mycompletion

\vspace{1em}
\noindent Laboratory: CSAIL 

\vspace{1em}
\noindent Brief Statement of the Problem:

\vspace{1em}
\begin{minipage}{\dimexpr\textwidth-1cm}

\noindent 
Users' privacy requirements for their sensitive data on web applications are
highly contextual and constantly in flux. For example, a user might wish to hide
and protect data of an e-commerce or dating app profile when inactive, but also
want their data to be present should they return to use the application. 
%
Today, however, services often provide only coarse-grained, blunt tools that
result in all-or-nothing exposure of users’ private information.
%
This thesis introduces the notion of \emph{disguised data}, a reversible state
of data in which sensitive data is selectively hidden.
%
This thesis then describes \sys---the first system for disguised data---which
helps web applications allow users to remove their data without permanently
losing their accounts, anonymize their old data, and selectively dissociate
personal data from public profiles.
%
\sys helps developers support these features while maintaining application
functionality and referential integrity.
%via \emph{disguising} and \emph{revealing} transformations.
%%
%Disguising selectively renders user data inaccessible via encryption, and
%revealing enables the user to restore their data to the application.
%%
%\sys's techniques allow transformations to compose in any order, \eg deleting a
%previously anonymized user's account, or restoring an account back to an
%anonymized state.
%

\end{minipage}

\vspace{1em}
\noindent Supervision Agreement:

\vspace{1em}
\begin{minipage}{\dimexpr\textwidth-1cm}
\noindent The program outlined in this proposal is adequate for a PhD thesis. The
supplies and facilities required are available, and I am willing to supervise
the research and evaluate the thesis report.
\end{minipage}

\vspace{2em}
\begin{flushright}
    \rule{0.7\linewidth}{0.1pt}\\
    {\normalsize Frans Kaashoek, Charles Piper Professor of EECS}\\
    {\normalsize Malte Schwarzkopf, Professor of Computer Science, Brown University}
\end{flushright}

\restoregeometry

\endgroup

\begin{comment}
\title{Helping Web Developers Give Users Control Over Their Data (Proposal)}
\author{Lillian Tsai}
\prevdegrees{
  A.B., Harvard University (2017) \\
  S.M., Harvard University (2017)}
\department{Department of Electrical Engineering and Computer Science}
\degree{Doctor of Philosophy}
\degreemonth{May}
\degreeyear{2024}
\thesisdate{May 13, 2024}
\supervisor{M. Frans Kaashoek}{Charles Piper Professor of Electrical Engineering and Computer Science}
\cosupervisor{Malte Schwarzkopf}{Professor of Computer Science, Brown University}
\chairman{Leslie A. Kolodziejski}{Professor of Electrical Engineering and
  Computer Science \\ Chair, Department Committee
  on Graduate Students}
\end{comment}
