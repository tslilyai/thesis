\iffalse
\begin{figure}[t]
    \centering
    \small
    \begin{tabular}{m{0.23\linewidth}|m{0.19\linewidth}|m{0.19\linewidth}|>{\RaggedRight\arraybackslash}m{0.19\linewidth}} %|m{0.08\linewidth}}
        \multirow{2}{*}{\centering\textbf{System}} &
            \multicolumn{3}{c}{\textbf{User $u$'s data is protected against...}}\\
        \cline{2-4}
            & \emph{SQL injection}
            & \emph{Compromised user $\neq u$}
            & \emph{Server compromise} \\
%            & \emph{Compromise of $u$} \\
        \hline
        Qapla~\cite{qapla} & \hfil \checkmark & & \\
        \hline
        CryptDB~\cite{cryptdb} & \hfil \checkmark & & \hfil \checkmark \\
        \hline
        \sys & \hfil \checkmark & \hfil \checkmark & \\
        \hline
        \syscrypt & \hfil \checkmark & \hfil \checkmark & \hfil \checkmark \\
    \end{tabular}
    \caption{Threats protected against by different classes of systems.}
    \label{tab:related_threats}
\end{figure}
\fi

This thesis describes \sys, the first system to address the problem of
reversible and composable data transformations for flexible data privacy in web
applications.
%
Existing systems aim instead to support data deletion, prevent unauthorized data
access, or protect against database server compromise---valuable, but
complementary goals to \sys's.

\textbf{Data deletion tools}, such as DELF at
Meta~\cite{delf} and K9DB\cite{k9db} help correctly delete data.
%
%DELF lets developers specify deletion policies via annotations on
%social graph edges and object types, and ensures correct cascading data
%deletion.
%
%Other systems support wholesale user data deletion by tracking
%data ownership by modifying the data layout~\cite{usershards, k9db} or
%tracking information flow~\cite{schengendb}.
%
While \sys also supports GDPR account deletion, \sys's focus is on more nuanced
use cases beyond simple deletion: \sys allows users to return after deletion,
hides old data for inactive users, or hides some but not all data so the user
can continue using the application.
%
\sys could, however, benefit from these systems' proposed techniques to track a
user's data.
%

\textbf{Policy enforcement systems} such as Qapla~\cite{qapla},
Blockaid~\cite{blockaid} and others~\cite{static, jeeves, jif, hails, ifdb,
oracle, multiverse, sieve} aim to prevent unauthorized access to data and
protect against leakage via compromised accounts or SQL injections. 
%They enforce developer-specified
%visibility and access control policies via information flow
%control~\cite{static, jeeves, jif, hails, ifdb}, authorized views~\cite{oracle},
%per-user views~\cite{multiverse}, or by blocking or rewriting database
%queries~\cite{blockaid, qapla, sieve}.
%
However, policy-enforcing systems do not help users anonymize data or maintain
application integrity constraints, which is \sys's explicit
goal.
%
Instead of denying access to data, \sys changes the database contents so sensitive
data is no longer available in the database, and thus unavailable even to the
service itself.
%

\textbf{Encrypted storage systems} such as CryptDB~\cite{cryptdb} and
Mylar~\cite{mylar} protect against database server compromise, with some
limitations~\cite{grubbs}.
%
%These systems encrypt data in the database, and
%ensure that only users with access to the right keys can decrypt the data.
%
%Applications must handle keys, and send queries
%either through trusted proxies that decrypt data~\cite{cryptdb}, or move
%application functionality client-side~\cite{mylar}.
%
Encrypted databases have orthogonal goals to \sys's: while they protect data at all times
against attackers who do not have the keys, encrypted databases do not help applications
anonymize or temporarily remove data, which \sys does.
%
Any user with legitimate access can view the data in an encrypted database,
whereas \sys removes \xxed data from the database.

\textbf{Other related work} uses crytographic or systems security methods to
prove enforcement of user-defined policies~\cite{zeph, riverbed,
vanish}\lyt{does vanish belong here?}, or explores
new decentralized paradigms for web application operation~\cite{solid, bstore,
databox, diy, amber, oort, w5, blockstack}.  These may restrict application
functionality, whereas \sys leaves the data and business models unchanged.
\lyt{ignoring single-device mobile dada encryption here.}
%
%%
%{Devices} using iOS~\cite{applesecurity},
%Android~\cite{applesecurity}, or CleanOS~\cite{cleanos} revoke
%data access via encryption, like \sys does.
%%
%However, these systems operate in settings that store only a single user's data;
%\sys instead tackles the problem of
%transformations that operate over multiple users' data and shared data without
%breaking the application.
%%
%
%%
%Vanish~\cite{vanish} provides users with self-destructing data and a {proof of
%data deletion} using decentralized infrastructure and cryptographic techniques
%(with limitations against Sybil attacks~\cite{defeat_vanish}). Unlike
%\sys, Vanish cannot restore deleted data and requires extensive application
%restructuring.
%%
%
%%
%{Sypse~\cite{sypse}} pseudonymizes user data and partitions
%personally identifying information (PII) from other data. Instead of
%partitioning data, \sys modifies the database and stores \xxed data
%encrypted.
%%
%
%{Decentralized platforms} such as Solid~\cite{solid}, BSTORE~\cite{bstore},
%Databox~\cite{databox}, and others~\cite{diy, amber, oort, w5, blockstack} put
%data directly under user control, since users store their own data.
%%
%But decentralized platforms burden users with maintaining infrastructure, lack
%the capacity for server-side compute, and break today's ad-based
%business model.
%%
%By contrast, \sys leaves the data and business models unchanged,
%and stores all data, including \xxed data, on the application's servers.
%%
%
%Some platforms can prove that server-side processing respects
%{user-defined policies} via cryptographic means~\cite{zeph} or
%systems security mechanisms~\cite{riverbed}. This may restrict feasible
%application functionality (\eg to additively homomorphic functions), or restrict
%combining data with different policies. \sys protects data only after
%\xxing, but allows unrestricted application functionality before
%\xxing.
