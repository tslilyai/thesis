\let\rwhighlight\relax

\begin{comment}
\begin{figure}[t]
    \centering
    \small
    \begin{tabular}{m{0.23\linewidth}|m{0.19\linewidth}|m{0.19\linewidth}|>{\RaggedRight\arraybackslash}m{0.19\linewidth}} %|m{0.08\linewidth}}
        \multirow{2}{*}{\centering\textbf{System}} &
            \multicolumn{3}{c}{\textbf{User $u$'s data is protected against...}}\\
        \cline{2-4}
            & \emph{SQL injection}
            & \emph{Compromised user $\neq u$}
            & \emph{Server compromise} \\
%            & \emph{Compromise of $u$} \\
        \hline
        Qapla~\cite{qapla} & \hfil \checkmark & & \\
        \hline
        CryptDB~\cite{cryptdb} & \hfil \checkmark & & \hfil \checkmark \\
%        \hline
%        \sys & \hfil \checkmark & \hfil \checkmark & \\
%        \hline
%        \syscrypt & \hfil \checkmark & \hfil \checkmark & \hfil \checkmark \\
    \end{tabular}
    \caption{Threats protected against by different classes of systems.}
    %\sys
    %also keeps the application functional if it relies on data
    %that users want to protect.}
    \label{tab:related_threats}
\end{figure}
\end{comment}

%
%\sys gives developers tools to remove and redact user data from a web
%application without breaking the application.
%
Existing systems for data protection aim to support data deletion, prevent
unauthorized data access, protect against database server compromise, or store
data in privacy-preserving ways.
%
However, none of these systems are explicitly designed to flexibly handle user
data (via fine-grained redactions), and to do so without breaking
the application. Instead, many of these systems provide orthogonal or partial
solutions.

\subsection{Data Deletion Tools.}
Laws and regulations such as the EU's General Data Protection Regulation
(GDPR)~\cite{eu:gdpr}, the California Consumer Privacy Act (CCPA)~\cite{ccpa},
and others~\cite{va:privacy-act, china:gdpr-like} give users the right to
request deletion of their data from web applications.
%
Data deletion tools developed at large organizations, such as DELF at
Meta~\cite{delf}, help to correctly delete data in response to such requests.
%
DELF helps developers specify correct deletion policies via annotations on
social graph edges and object types, and ensures correct cascading data
deletion.


Other systems to help with wholesale user data deletion modify the data layout~\cite{usershards} or use fine-grained
information flow tracking~\cite{schengendb} to track the ownership and
provenance of database rows.
%
This tracks how information propagates and provides information to support
wholesale user data deletion.

%
However, these data deletion systems target only simple deletion, and do not address
the entire range of possible user data controls (\eg temporarily removing old data of
inactive users). 

%%
%Like data deletion systems, \sys supports GDPR account deletion.
%However, \sys also supports use cases beyond simple deletion.
%
%\sys can seal data in a GDPR-compliant manner and allow the user to return,
%seal old data for inactive users, or seal some but not all data,
%allowing the user to continue using the application.
%
%Finally, \sys emphasizes maintaining application functionality, which
%involves storing placeholder data and fake users in the database.
%

%
\subsection{Protecting Against Unauthorized Access} 
\textbf{Policy enforcement systems} such as Qapla~\cite{qapla}
aim to prevent
unauthorized access to data and protect against leakage via compromised,
unauthorized accounts or SQL injections.
%
They enforce developer-specified visibility and access control policies via
information flow control~\cite{static, jeeves, jif, hails, ifdb}, authorized
views~\cite{oracle} or per-user views~\cite{multiverse}, or blocking or
rewriting database queries~\cite{blockaid, qapla, sieve}.
%
%These systems restrict access to user data by restricting a client's queries
%depending on the access policy of the client's application role.
%
%Users can only successfully query data if authorized by their application
%role's policy.% to view certain data can successfully query it.
%
%This addresses the threat of a compromised \emph{unauthorized} user account, as
%well as arbitrary SQL queries via SQL injections: in either case, the exposed
%data is restricted to what the policy allows.
%
%However, if a compromised user (\eg an admin) has a legitimate reason to view
%data and their policy authorizes broad access, policy enforcement will fail
%to contain the damage.
%

%
Some platforms can prove that server-side processing respects
user-defined data policies via cryptographic means~\cite{zeph} or
systems security mechanisms~\cite{riverbed}.
%
This may restrict feasible application functionality (\eg to additively
homomorphic functions), or restrict combining data with different policies.
%
%\sys only protects user data after sealing, but in exchange allows unrestricted
%application functionality before sealing.
%

\textbf{Encrypted storage systems} like
CryptDB~\cite{cryptdb} and
Mylar~\cite{mylar} aim to protect against database server compromise (with some
limitations~\cite{grubbs}), as well as SQL injection attacks.
%
These systems encrypt data in the database, and ensure that only
users with legitimate reasons to view the data (\ie the owner or an
admin) can decrypt the data.
%
Applications handle keys, and send queries either through
trusted proxies that decrypt data~\cite{cryptdb}, or move application
functionality client-side~\cite{mylar}.
%
%However, they fail to prevent \emph{authorized} accesses from a compromised
%account that (for legitimate purposes) holds the key material to decrypt the
%data.

Policy-enforcing and encrypted storage systems do not help users redact data,
nor maintain application functionality in the face of partially or fully
redacted data. 
%
Their goal is orthogonal: to protect unauthorized data access via access control
mechanisms (encryption or policy checks).
%
These systems also keep data available in the database, thus leaving user data
vulnerable to attacks by compromised accounts with authorized access (\eg an
admin).

%
\subsection{New Database Designs}
Other work has created new privacy-focused data storage designs for applications.
Sypse~\cite{sypse} pseudonymizes user data and partitions personally-identifying
information (PII) from other data.
%
%Instead of partitioning data, \sys actually modifies the application database
%and stores sealed data encryptedly.
%
However, Sypse does not securely store (\eg with encryption) PII, nor modify the
actual data stored by the application
%, and thus tackles a different threat model.

%
Decentralized platforms such as Solid~\cite{solid}, BSTORE~\cite{bstore},
Databox~\cite{databox}, and others~\cite{diy, amber, oort, w5, blockstack} put
data fully and directly under user control, since users store their own data.
%
But this burdens users with maintaining infrastructure, and decentralized platforms
lack the capacity for server-side compute and break today's ad-based
business model.
%
%By contrast, \sys leaves the data and business models unchanged,
%storing all data, including sealed data, on the application's servers.
%


