%
To tackle the problem of flexible privacy in web applications, we present a
system design that moves closer to an internet where users can leave services
and return at any time, where old data on servers is protected by default, and
where services provide users with control over their identifying data visible to
the service and other users.

%
Our approach is to create a general system that helps developers specify and
apply two kinds of transformations: \emph{\xxing transformations}, which render
all or some of the user’s original sensitive data inaccessible to the
application; and \emph{revealing transformations}, which restore the original
data at a user’s request.
%
\Xxing transformations aim to protect the confidentiality of users' \xxed data
(\eg links to throwaway accounts or old HotCRP reviews) even if the application
is later compromised (\eg via a SQL injection or a compromised admin's account).
%

%
We demonstrate our approach in \sys, a system that realizes disguising and
revealing transformations for database-backed web applications via a set of
primitives that have well-defined semantics and compose cleanly.
%
Developers specify the transformations that their application should provide,
and \sys takes care of correctly applying, composing, and optionally reverting
them, while maintaining application functionality and referential integrity.
%

%
\subsection{Challenges}
%
We had to address three challenges to make this approach work.
%
First, \sys needs to present a simple, yet versatile interface for developers to
specify \xxing transformations.
%
\sys addresses this challenge with a restricted programming model centered
around three primitives: remove, modify, and decorrelate (which reassigns data
to placeholder users).
%
This model limits the potential for developer error, and lets \sys derive the
correct \xxing and revealing operations, while supporting a wide range of
transformations.
%

%
Second, to work with existing applications in practice, \sys's \xxing
transformations should require minimal application modifications.
%
To achieve this, \sys introduces \emph{pseudoprincipals}, anonymous placeholder
users that are inserted into the database on \xxing and exist solely to own data
decorrelated from real users (\eg because the application requires the data to
continue operating) and maintain referential integrity.
%
Pseudoprincipals can also act as built-in ``throwaway accounts,'' as they let
the user disown data after-the-fact, as well as potentially later reassociate
with it.
%
To correctly reason about ownership when data may be decorrelated multiple times
(\eg by global anonymization after throwaways have been created), \sys maintains
an encrypted speaks-for chain of pseudoprincipals that only the original user
can unlock and modify.
%

%
Third, \sys needs to have access to the original data for users to be able to
reveal their data and return to the application, but the whole point is to make
that data inaccessible to the service.
%
While \sys could ask users to store their own \xxed data, this would be
burdensome.
%
Instead, \sys stores the \xxed data on the server in encrypted form, and unlocks
and restores data to the service only when a user provides their credentials to
reveal.
%
