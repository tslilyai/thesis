\section{Motivation}
\label{sec:intro:motivation}
Users today have little to no control over the data they give to web
applications.  They must either avoid using an application, or leave their data
susceptible to leaks, \eg via SQL injections or compromise of other users'
accounts.  Incentivized by growing public demand and laws such as the GDPR and
CCPA~\cite{eu:gdpr, ccpa}, developers increasingly need to implement mechanisms
to give users control over their data’s exposure.

Central to this conversation are two needs: users want to protect their data,
and applications require data to be available in order to function. Problems
arise when application functionality requires the same data users want to
protect: for example, deleting a post may orphan other users' comments on it,
crashing the application when it expects comments to have an associated post.
Handling this correctly is difficult for developers---naively removing user data
can break the application or reduce its functionality. Removal is also usually
permanent, hurting both users and applications: once a user deletes their data,
there is no coming back.

Large companies have built systems to help developers get data deletion
right~\cite{delf}, but the problem goes beyond data deletion. Users may want to
take a break from using an application, but rest easy that their data is
protected while they’re inactive; or users may want more fine-grained
control—\ie have their data exist in a state where it is both visible and
protected. For example, a user might want to disassociate content with a
specific hashtag from a social media account. Because applications today lack
support for this, users resort to ad-hoc alternatives like throwaway accounts
and ``finstas''~\cite{reddit:throwaway, nytimes:finsta} to post content that is
visible but disconnected from the ``real'' account. This burdens users with
managing multiple accounts and requires planning ahead, since moving posts into
a throwaway account or reclaiming them is impossible.

\section{Approach}
\label{sec:intro:approach}
This thesis investigates novel systems to let web application developers provide users with
means to manage their data in the gray space where users' need for data
protection and application functionality requirements overlap.  

We start by building Edna, a library that integrates with web applications and
provides abstractions that let developers specify not just wholesale removal of
data, but also flexible redaction or decorrelation of data. Edna then changes
the database contents in well-defined ways that avoid breaking the application.
Developers and users both benefit from Edna: applications can expose and
advertise new privacy-protection features and reduce their exposure in the event
of a data breach, while users gain more control over their sealed data,
including the convenience of returning to the application as if their data had
been there all along.

Edna introduces \emph{sealing}, which removes or redacts some or all of a user's
data, and \emph{revealing}, which restores the sealed data at a user's request.
Sealed data remains on the server, but is encrypted and inaccessible to the web
application. Sealing changes the database contents and replaces the data to seal
with placeholder values where necessary (\eg comments require an associated
post). 

We designed Edna's seal and reveal abstractions to flexibly meet applications'
diverse needs while still protecting sensitive user data. Edna provides
developers with three well-defined primitives for anonymizing data: remove,
modify, and decorrelate. Developers compose these to create \emph{seal
specifications}, which describe the data to seal and which primitives to apply.
When invoked with a seal specification, Edna changes the database contents as
specified: ``remove'' drops rows, ``modify'' changes the contents of specific
cells, and ``decorrelate'' introduces \emph{pseudoprincipals}, anonymous
placeholder users. These pseudoprincipals help maintain referential integrity,
but also can act as built-in ``throwaway accounts.'' Rather than users having to
create and maintain throwaways themselves, Edna's pseudoprincipals enable the
user to later reassociate with their throwaway's data via revealing, or create
throwaways to disown posts after-the-fact.

\begin{comment}
\section{Related Work}
\label{sec:intro:related}

\section{Approach}
\label{sec:intro:approach}


\section{Contributions}
\label{sec:intro:contributions}


\section{Reading Guide}
\label{sec:intro:reading-guide}
\end{comment}
