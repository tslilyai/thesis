\section{Motivation}
\label{sec:intro:motivation}
Users today have little to no control over the data they give to web
applications.  They must either avoid using an application, or leave their data
susceptible to leaks, \eg via SQL injections or compromise of other users'
accounts.  Incentivized by growing public demand and laws such as the GDPR and
CCPA~\cite{eu:gdpr, ccpa}, developers increasingly need to implement mechanisms
to give users control over their data’s exposure.

Central to this conversation are two needs: users want to protect their data,
and applications require data to be available in order to function. Problems
arise when application functionality requires the same data users want to
protect: for example, deleting a post may orphan other users' comments on it,
crashing the application when it expects comments to have an associated post.
Handling this correctly is difficult for developers---naively removing user data
can break the application or reduce its functionality. Removal is also usually
permanent, hurting both users and applications: once a user deletes their data,
there is no coming back.

Large companies have built systems to help developers get data deletion
right~\cite{delf}, but the problem goes beyond data deletion. Users may want to
take a break from using an application, but rest easy that their data is
protected while they’re inactive; or users may want more fine-grained
control—\ie have their data exist in a state where it is both visible and
protected. For example, a user might want to disassociate content with a
specific hashtag from a social media account. Because applications today lack
support for this, users resort to ad-hoc alternatives like throwaway accounts
and ``finstas''~\cite{reddit:throwaway, nytimes:finsta} to post content that is
visible but disconnected from the ``real'' account. This burdens users with
managing multiple accounts and requires planning ahead, since moving posts into
a throwaway account or reclaiming them is impossible.

\section{Approach}
\label{sec:intro:approach}
This thesis investigates novel systems to let web application developers provide
users with means to manage their data in the gray space where users' need for
data protection and application functionality requirements overlap.  

We start by building Edna, a library that integrates with web applications and
provides abstractions that let developers specify not just wholesale removal of
data, but also flexible redaction or decorrelation of data. Edna then changes
the database contents in well-defined ways that avoid breaking the application.
Developers and users both benefit from Edna: applications can expose and
advertise new privacy-protection features and reduce their exposure in the event
of a data breach, while users gain more control over their sealed data,
including the convenience of returning to the application as if their data had
been there all along.

We then propose Hydra, a library that enables web applications to provide their
users inbuilt for splintering and merging their web identities within the
application. With Hydra, applications can let users decorrelate data to
context-specific identities (\eg an identity for work or for family), easily
move their data between identities, spin off new identities when necessary, and
merge identities if desired. Importantly, only the user can determine which
web application identities are linked to themselves.
%
Hydra explores the result of taking application-supported data
decorrelation to the extreme, while preserving application functionality and
protecting correlations between data and users. Applications both retain
(decorrelated) data for \eg analytical or advertising purposes, while users gain
security and control over what data is linked to which of their web application
identities.
%implements decorrelation of data, similar 

\begin{comment}
\section{Related Work}
\label{sec:intro:related}

\section{Approach}
\label{sec:intro:approach}


\section{Contributions}
\label{sec:intro:contributions}


\section{Reading Guide}
\label{sec:intro:reading-guide}
\end{comment}
