\section{Introduction}
The question of who controls personal data, and how it is used, has increasingly
surfaced in the web ecosystem. 
%
Recent legal developments such as the GDPR and CCPA, as well as growing public
demand and news coverage of \eg data breaches, have increased the need for
better user data handling in web applications. 
%
%Users today give up control of their data to web services in order to use these
%services, which can put user data at risk of data breaches and expose it to
%third parties in ways unwelcome or unknown to the user.
%developers increasingly need to implement better user controls in
%their web applications.
%However, doing so is challenging: the data that users want to protect is often
%important for the service's function.
%
This thesis identifies key challenges that web developers face today in
implementing better user data controls, and contributes practical systems that
can help to give control of user data in web services back to the user.
%
In particular, this thesis introduces a novel system design that allows
applications to implement a variety of user data controls while still remaining
functional.
%tackles the gray space where users’ need for data protection and
%application functionality requirements overlap. 

\section{Motivation}
\label{sec:intro:motivation}
Users today have little to no control over the data they give to web
applications.  They must either avoid using an application, or leave their data
susceptible to leaks, \eg via SQL injections or compromise of other users'
accounts.  Incentivized by growing public demand and laws such as the GDPR and
CCPA~\cite{eu:gdpr, ccpa}, developers increasingly need to implement mechanisms
to give users control over their data’s exposure.

Central to this conversation are two needs: users want to protect their data,
and applications require data to be available in order to function. Problems
arise when application functionality requires the same data users want to
protect: for example, deleting a post may orphan other users' comments on it,
crashing the application when it expects comments to have an associated post.
Handling this correctly is difficult for developers---naively removing user data
can break the application or reduce its functionality. Removal is also usually
permanent, hurting both users and applications: once a user deletes their data,
there is no coming back.

Large companies have built systems to help developers get data deletion
right~\cite{delf}, but the problem goes beyond data deletion. Users may want to
take a break from using an application, but rest easy that their data is
protected while they’re inactive; or users may want more fine-grained
control—\ie have their data exist in a state where it is both visible and
protected. For example, a user might want to disassociate content with a
specific hashtag from a social media account. Because applications today lack
support for this, users resort to ad-hoc alternatives like throwaway accounts
and ``finstas''~\cite{reddit:throwaway, nytimes:finsta} to post content that is
visible but disconnected from the ``real'' account. This burdens users with
managing multiple accounts and requires planning ahead, since moving posts into
a throwaway account or reclaiming them is impossible.

These problems call for research into systems that let web application
developers provide users with means to manage their data in the gray space where
users' need for data protection and application functionality requirements
overlap. 
