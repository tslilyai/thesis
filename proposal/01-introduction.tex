Users' privacy requirements for their sensitive data on web applications are
highly contextual and constantly in flux. For example, a user might wish to hide
and protect data of an e-commerce or dating app profile when inactive, but also
want their data to be present should they return to use the application. 
%
%But today, achieving this flexible privacy remains out of reach, in part
%because of the many challenges facing web developers in building services that
%can adapt to changing privacy needs. 
Today, however, services often provide only coarse-grained, blunt tools that
result in all-or-nothing exposure of users’ private information.
%

%
This thesis introduces the notion of \emph{disguised data}, a reversible state
of data in which sensitive data is selectively hidden.
%
This thesis then describes \sys---the first system for disguised data---which
helps web applications allow users to remove their data without permanently
losing their accounts, anonymize their old data, and selectively dissociate
personal data from public profiles.
%
\sys helps developers support these features while maintaining application
functionality and referential integrity via \emph{disguising} and \emph{revealing}
transformations.
%
Disguising selectively renders user data inaccessible via encryption, and
revealing enables the user to restore their data to the application.
%
\sys's techniques allow transformations to compose in any order, \eg deleting a
previously anonymized user's account, or restoring an account back to an
anonymized state.
%

\subsection{Motivation} 
Many users today have tens to hundreds of accounts with web
services that store sensitive data, from social media to tax preparation and
e-commerce sites~\cite{tens,hundreds,password_life_cycle}.
%
And while users now have the right to delete their data (via \eg the
GDPR~\cite{eu:gdpr} or CCPA~\cite{ccpa}), users want and deserve more nuanced
controls over their data that don't exist today.
%

Consider Twitter: after a change in management~\cite{musk-twitter}, many users
wanted to leave the platform and try out alternatives (\eg Mastodon).  But each
user faced a tricky question: should they keep their Twitter account, or should
they delete it?
%
Advice on how to quit Twitter~\cite{quit-twitter-india, quit-twitter-mash}
highlight how keeping an inactive account leaves sensitive information (\eg
private messages) vulnerable on Twitter's servers; but deleting the account
prevents the user from changing their mind and coming back.  Hence, many users
left Twitter but kept their accounts~\cite{nbc-twitter,shondarhimes,kenolin}.
%
A better solution would let users temporarily revoke Twitter's access to their
data while having the option to come back.

Similarly, users give dating apps personal data, and frequently deactivate and
reactivate their accounts.
%
This sensitive data should be protected from the application and potential data
breaches~\cite{tinder, okcupid} when a user deactivates their account, but be
readily available when they choose to return.

Users may also prefer old data, such as past purchases in an online store or
their passport details with a hotel, to be inaccessible to the service after
some time of inactivity, and therefore protected from leaks or service
compromises~\cite{retention,breach:marriott}. Or users may prefer
to---explicitly or automatically---dissociate their identity from old data, such
as teenage social media posts or old reviews on HotCRP.
%
Today, users work around the lack of such support by explicitly maintaining
multiple identities (\eg Reddit throwaway
accounts~\cite{reddit:throwaway} and Instagram
``finstas''~\cite{nytimes:finsta}), an inflexible and laborious solution.
%

%
Providing this functionality can benefit both the service and the user.
%
It helps the service comply with privacy regulations, reduces its liability on
data breaches, and appeals to privacy-conscious users; meanwhile, the user can
rest assured that their privacy is protected, but can also get their data back
and reveal their association with it if they want.
%
This thesis introduces the new abstraction of \emph{disguised data} that
enables this functionality and supports these use cases. 

\subsection{Why is this hard?} 
Today, flexible privacy remains out of reach for users of
web applications in part because getting it right is hard. 
%
Real applications have complex notions of privacy, data ownership, and data
sharing.
%
Simple solutions that \eg delete all data associated with a user can break
referential integrity or create orphaned data, which requires application
changes to handle correctly, and lack support for users to return.
%
To solve this manually, a developer would have to carefully perform
application-specific database changes to remove data, store any data removed to
be able to later restore it, and correctly revert the database changes on
restoring.
%
Furthermore, stored data must be inaccessible to the application and protected
against data breaches, but must be accessible if the user chooses to return.
%

Developers would also have to reason about interactions between multiple
data-redacting features.
%
For example, imagine an application that supports both account deletion and
anonymizing old data: if a user wants to delete all their posts after they have
been anonymized, a SQL query must somehow determine which anonymized posts
belong to the user in order to remove them.
%
And if the user later wants to return, the developer must account for the
applied anonymization and restore posts as anonymized.
%
