\label{sec:intro:contrib}
%
This thesis makes the following contributions: 
\begin{enumerate}[nosep]
    \item The identification and exploration of the problem of \emph{flexible
        privacy} for user data in web applications. \lyt{including a survey of 
        state of the art systems for data privacy and data protections?}

    \item Abstractions to achieve flexible privacy in web applications via \emph{disguised data},
        including abstractions for \xxing and revealing, and a small set of
        data-anonymizing primitives (remove, modify, decorrelate) that cover a
        wide range of application needs and compose cleanly.

    \item Techniques to implement these abstractions, including
    pseudoprincipals, speaks-for and diff
    records, and speaks-for chains.

    \item A design and realization of these techniques in \sys, a prototype
        library that implements user data control via disguising and revealing.

    \item Case studies that integrate \sys with three real-world web
    applications and demonstrate \sys's ability to enable composable and
    reversible transformations.

    \item An evaluation of \sys's effectiveness and performance, including how
    \sys contrasts with and complements related work (Qapla~\cite{qapla} and
    CryptDB~\cite{cryptdb}).  
    
\item \lyt{An exploration of additional areas to apply flexible privacy? (Dunno
    if this is a contribution, but it would be nice to demonstrate how Funhouse
        fits into the picture!)}
\end{enumerate}
%
\S\ref{s:edna} appends \sys's paper describing \sys's design, implementation,
and evaluation to this proposal.

\subsection{Limitations}
%
While \sys enables \xxing and revealing transformations in a broad class of
applications, \sys has some limitations.
%
First, \sys assumes bug-free \xx specifications, and that applications use \sys
correctly.
%
Second, while \sys helps developers add user data controls to single applications,
\sys does not tackle the problem of data sharing between services.
%
Third, \sys does not aim to protect un\xxed data in the database against compromise;
combining \sys with an encrypted database can add this protection.
%
Finally, attacks to identify users from \sys's metadata (\eg the size of
stored \xxed data) or placeholder data left in the database (\eg embedded text)
are out of scope.
%

\subsection{Extensions: Robust Revealing}
\sys currently performs a set of consistency checks that allow it to detect if
revealing transformations will violate referential integrity or other structual
database invariants (\eg uniqueness requirements). However, because disguised
data remained inaccessible to the application prior to reveal, application
updates that implicitly enforce invariants on application data remain unknown
(and thus uncheckable) by \sys.
%
Futhermore, \sys's checks are conservative: data could still be revealed even
in the presence of structual database transformations like schema migrations, if
\sys transforms the disguised data in appropriate ways prior to reveal.

These extensions will add the following contributions:
\begin{enumerate}[nosep]
    \item Support for revealing in the presence of schema migrations that affect
        the data to reveal.
    \item Addition of implicit application invariants to \sys's consistency
        checks, which enable \sys to prevent revealing transformations from
        breaking implicit application semantics.
\end{enumerate}
