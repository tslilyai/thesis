\label{sec:intro:contrib}
Edna provides the following contributions:
\begin{enumerate}[nosep] 
    \item A new paradigm for developers to provide flexible user control over data in web applications, with cryptographic protection and while maintaining application functionality and invariants.
    \item Abstractions for sealing and revealing, and a small set of data-anonymizing primitives (remove, modify, decorrelate) that cover a wide range of application needs and compose cleanly.
    \item The design and implementation of Edna, our prototype library that implements user data control via sealing and revealing, as well as Edna-CryptDB, which additionally encrypts unsealed data.
    \item Case studies of integrating Edna with three real-world web applications, an evaluation of Edna's effectiveness, performance, and security, and a comparison to Qapla.
\end{enumerate}

\begin{comment}
\subsection{Design}
We combine Edna with an encrypted database to achieve stronger guarantees.
Edna-CryptDB simultaneously protects against database server compromises even for
unsealed data and adds Edna's protections for sealed data to encrypted
databases, which have no built-in support for removing sensitive data without
breaking the application.

\subsection{Evaluation}
To investigate the need for Edna as a new system, we tried to realize Edna's
functionality atop Qapla~\cite{qapla}, a framework that rewrites SQL queries to
conform to access control policies. We found that Qapla requires invasive
application changes, its abstractions are awkward for sealing and revealing, and
Qapla's query rewriting slows down common queries.
\end{comment}

\subsection{Limitations}
Edna has some limitations. Its usability goals, and specifically the fact that
placeholder data continues to exist in the database, prevent Edna from
protecting against inference attacks (\eg statistical correlation attacks). Edna
also assumes bug-free seal specifications, and that applications use
Edna correctly. Finally, Edna provides limited protection for sealed data
against attackers who compromise the database server, as auxiliary database
information (\eg logs) may contain old plaintext data; Edna-CryptDB strengthens
these protections.

\subsection{Extensions: Hydra}
Hydra narrows down on a specific aspect of Edna, namely how users control their
identities and links between their identities and data in web applications.
%Users today will manually implement ad-hoc alternatives like throwaway accounts
%and ``finstas''~\cite{reddit:throwaway, nytimes:finsta} to post content that is
%visible but disconnected from the ``real'' account. 
%
Hydra provides a library that enables web applications to provide their
users inbuilt for splintering and merging their web identities within the
application. With Hydra, applications can let users decorrelate data to
context-specific identities (\eg for work or for family), easily
move their data between identities, spin off new identities when necessary, and
merge identities if desired. Importantly, only the user can determine which
web application identities are linked to themselves.
%
Hydra explores the result of taking application-supported data
decorrelation to the extreme, while preserving application functionality and
protecting correlations between data and users. 
%Applications both retain
%(decorrelated) data for \eg analytical or advertising purposes, while users gain
%security and control over what data is linked to which of their web application
%identities.

Hydra adds the additional contributions:
\begin{enumerate}[nosep] 
    \item A new paradigm for developers to provide user account splintering and
        merging, while cryptographically protecting links between user accounts
        and while maintaining application functionality and invariants.
    \item A case study integrating Hydra with a real-world web application,
        Lobsters, and an evaluation of Hydra's effectiveness, performance, and
        security.
\end{enumerate}

%Multi-account containers in Firefox let users act as distinct identities as they
%browse the web.
