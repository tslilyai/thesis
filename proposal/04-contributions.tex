\label{sec:intro:contrib}
%
This thesis makes the following contributions: 
\begin{enumerate}[nosep]
    \item The identification and exploration of the problem of \emph{flexible
        privacy} for user data in web applications.

    \item Abstractions to achieve flexible privacy in web applications via \emph{disguised data},
        including abstractions for \xxing and revealing, and a small set of
        data-anonymizing primitives (remove, modify, decorrelate) that cover a
        wide range of application needs and compose cleanly.

    \item Techniques to implement these abstractions, including
    pseudoprincipals, speaks-for and diff
    records, and speaks-for chains.

    \item A design and realization of these techniques in \sys, a prototype
        library that implements user data control via disguising and revealing.

    \item Case studies that integrate \sys with three real-world web
    applications and demonstrate \sys's ability to enable composable and
    reversible transformations.

    \item An evaluation of \sys's effectiveness and performance, including how
    \sys contrasts with and complements related work (Qapla~\cite{qapla} and
    CryptDB~\cite{cryptdb}).  
\end{enumerate}
%
Appended to this proposal is the \sys paper describing \sys's design, implementation, and evaluation.

\iffalse
\subsection{Limitations}
%
While disguised data can help developers to add more flexible user data
controls, the abstraction and its implementation in \sys have some limitations. 
%
First, disguised data is a concept scoped for single applications, and does not tackle the problem of data sharing between services.
%
%While \sys enables \xxing and revealing transformations in a broad class of
%applications, \sys has some limitations.
%
Second, \sys assumes bug-free \xx specifications, and that applications use \sys
correctly.
%
Third, \sys does not aim to protect un\xxed data in the database against compromise;
combining \sys with an encrypted database can add this protection.
%
Finally, attacks to identify users from \sys's metadata (\eg the size of
stored \xxed data) or placeholder data left in the database (\eg embedded text)
are out of scope.
%
\fi

\subsection{Remaining Work: Revealing with Schema Migrations and Application
Updates}

In \sys's current design, \textbf{application updates} that implicitly enforce
invariants on application data remain unknown (and thus uncheckable) by \sys.
\sys also fails to reveal disguised data affected by \textbf{schema migrations}
performed since the time of disguise.
%
We will add an API for developers to log important updates and schema migrations with \sys, which \sys will apply to disguised data prior to restoring it to the database.

\sys will apply these updates or schema migrations prior to performing \sys's
existing consistency checks; these checks enable \sys to detect if revealing
transformations will violate referential integrity or other structual database
invariants (\eg uniqueness requirements). 
