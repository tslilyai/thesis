%%%%%%%%%%%%%%%%%%%%%%%%%%%%%%%%%%%%%%%%%%%%%%%%%%%%%%%%%%%%%%%%%%%%%%%%%%%%%%%%%%%%%%%%%%
This section evaluates \sys's ability to add new data-redacting features to
several applications; \S\ref{s:eval} evaluates the effort needed to do so and the
resulting performance.
%

%
We add \xxing and revealing transformations based on
the motivating examples in \S\ref{s:intro}
to three applications---Lobsters~\cite{lobsters},
WebSubmit~\cite{websubmit-rs},
and HotCRP~\cite{hotcrp}.
%using \sys's API and \sys's default semantics.
%and fully integrated \sys with Lobsters and WebSubmit.
%
%We wrote \xxing transformations for three applications---%
%Lobsters~\cite{lobsters},
%WebSubmit~\cite{websubmit-rs},
%and HotCRP~\cite{hotcrp}---%
%and fully integrated \sys with Lobsters and WebSubmit.
%
%These case studies show that \sys's abstractions and high-level API sufficiently
%support fine-grained, selective \xxing transformations necessary that add
%desirable application-specific \xxing support for users.
%


%%%%%%%%%%%%%%%%%%%%%%%%%%%%%%%%%%%%%%%%%%%%%%%%%%%%%%
\section{Lobsters}
\label{s:cs:lobsters}

%
Lobsters is a Ruby-on-Rails application backed by a MySQL database.
%
Beyond the previously-mentioned stories, tags, etc., Lobsters also contains
moderations that mark inappropriate content as removed.
%a link-sharing and discussion platform with 14.5k users~\cite{lobsters},
%
We added three \xxing transformations: account deletion with return; account
decay, \ie automatic dissociation and protection of old data; and topic-specific
throwaway accounts.
%

%%%%%%%%%%%%%%%%%%%%
%
\textbf{GDPR-compliant account deletion}
\one{} removes the user account;
\two{} removes information that's only relevant to the individual user, such as
their saved stories;
\three{} modifies story and comment content to ``[deleted content]'';
%, but retains the
%database objects in order to preserve comment threads (and nested threads) on stories;
\four{} decorrelates private messages; and
%if the other party is still active, and removes them otherwise; and
\five{} decorrelates votes, stories, comments, and moderations on the user's
data. % by associating each remaining object with a unique pseudoprincipal.
%
%The originals of all deleted and modified data are stored in \sys as encrypted
%diffs.
%
This preserves application semantics for other users---\eg vote counts remain
consistent even after account deletion, and other users' comments
remain visible---while protecting the privacy of removed users.
%
Important information such as moderations on user content remains in the
database, and \sys recorrelates it if the user restores their account.
%
After \sys applies the \xxing transformation, Lobsters emails the user a URL
that embeds the \xx ID.
%
The user can visit this URL and provide their credentials to restore their account.
%
%This transformation achieves GDPR compliance, as neither the application nor
%\sys have the cryptographic keys to read the \xxed data~\cite{gdpr-encryption},
%but users can return.
%

%%%%%%%%%%%%%%%%%%%%
The \textbf{account decay} transformation protects
user data after a period of user inactivity.
%
We added a cron job that applies account decay to
user accounts that have been inactive for over a year.
%
This
\one{} removes the user's account;
\two{} removes information only relevant to the user, such as saved stories;
\three{} and decorrelates votes, stories, comments, and moderations on the
user's data by associating them with pseudoprincipals.
%
%Removed data is stored as encrypted diffs.
%
%While similar to account removal, account decay does not remove story or comment
%content.
%
%This preserves greater utility for other users, but also potentially retains
%identifying information within the content.
%
Lobsters sends the user an email which informs them that their data has
decayed, and includes a URL with an embedded \xx ID that can reactivate or
completely remove the account if credentials are provided.
%

%%%%%%%%%%%%%%%%%%%%
Finally, topic-based throwaway accounts via \textbf{topic-based
anonymization}
enable users to decorrelate their content relating to a particular topic.
%
As per \S\ref{s:design:lobsters}, this \xxs contributions associated with the
specified tag by \one{} decorrelating tagged stories and comments associated with
tagged stories, and \two{} removing votes for tagged stories.
%
Again, Lobsters sends the user an email with links that allow reclaiming or
editing these contributions.
%
%Category-based content anonymization leaves the original user account active.
%However, as a consequence of anonymizing tag-related contributions, account
%statistics will not reflect the votes these contributions accumulated.

%
With \sys and its support for composing \xxing transformations, users can
delete accounts that have been decayed or dissociated into throwaways, and
can later reveal them.

%Importantly, with \sys, these transformations and users' revealing of them can occur in any
%order: \eg users can delete accounts that have been decayed or that have been
%dissociated into throwaways, and reveal deleted accounts.
%
% Furthermore, multiple users can perform account deletion with
% shared private messages, and \sys enforces the shared-data semantics in
% \S\ref{s:shared}.

\subsection{Application Updates and Schema Migrations}
\label{s:casestudies:updates}
We also added support for three recent application updates and schema
migrations applied by the Lobsters developers in 2023-2024. 

The first is an application update and schema migration which performs url
normalization~\cite{urinorm} on the \texttt{url} column of all rows in the
\texttt{stories} table, and then stores the normalized url text in a new
\texttt{normalized\_url} column.
%

%
The second changes the schema by adding a \texttt{show\_email} attribute to the
\texttt{users} table automatically set to \texttt{false}. 

%
Finally, the third creates a new \texttt{story\_texts} table which stores the
\texttt{title}, \texttt{description}, and \texttt{story\_cache} attributes of
rows from \texttt{stories}. This enables search over story contenct. The
\texttt{story\_cache} column is then removed from \texttt{stories}.

%
In addition to these updates, we inspected 20 other updates/migrations from the past
three years broadly classified into:
\begin{itemize}[nosep]
\item creating tables; 
\item removing, adding, or renaming table columns;
\item removing or adding indexes on new (or old) table columns; 
\item setting column constraints (\eg \texttt{NOT NULL}); 
\item modifying column content based on deterministic functions (\eg URL normalization); 
\item generating new rows for various tables based on existing table rows (\eg adding \texttt{story\_texts}  
when restructuring \texttt{stories}); and
\item updating a row based on the current state of other database tables (\eg updating
the \texttt{vote\_count} of comments based on the number of votes).
\end{itemize}
%
\sys can support all but the last category of updates, unless the update is
independent of the currency of the database state. As described in
\S\ref{s:design:updates:limitations}, unless it is valid application behavior for \eg
\texttt{vote\_count} to be 0 regardless of the number of votes, 
\sys currently will not correctly apply an update that sets the count of votes
for comments.  This last category represented 2 of the 20 inspected updates.
    However, we observe that these particular updates happen to be idempotent
    for any rows affected, and thus could also be reapplied after all data is
        revealed.
%

%
With \sys, users can still reveal their data disguised prior to any of these
migrations, as if the migration had occurred with their data present in the
database.


%%%%%%%%%%%%%%%%%%%%%%%%%%%%%%%%%%%%%%%%%%%%%%%%%%%%%%
\section{WebSubmit}
\label{s:case-websubmit}

%
We integrated \sys as a Rust library with WebSubmit~\cite{websubmit-rs}.
%
WebSubmit is a homework submission application used at Brown University, and
its schema consists of tables for lectures, questions, answers, and user accounts.
%
Clients create an account, submit homework answers, and view their submissions;
course staff can also view submissions, and add/edit questions and lectures.
%
The original WebSubmit retains all user data forever.
%
We added support for two \xxing
transformations: {GDPR-compliant user account removal} with return, and
{instructor-initiated answer anonymization}, which protects data of prior years'
students by decorrelating student answers for a given course.
%(associating one pseudoprincipal with a student's set of lecture answers).
%
These transformations allow instructors to retain FERPA-compliant~\cite{ferpa}
answers after the class has finished. %\hmng{each or both together?}
%
With \sys, students can delete their
accounts or access and view their answers even after class anonymization,
and can always restore their deleted accounts, including restoring them to
anonymized state.
%
%Our modified WebSubmit sends users emails with links to restore their account
%or edit their \xxed data.
%


%%%%%%%%%%%%%%%%%%%%%%%%%%%%%%%%%%%%%%%%%%%%%%%%%%%%%%%%%%%%%%%%%%%%%%%%%%%%%%%%%%%%%%%%%%
\section{HotCRP}
\label{s:case-hotcrp}

%
HotCRP is a conference management application whose users can be reviewers and/or
authors.
%
HotCRP's schema contains papers, reviews, comments, tags,
and per-user data such as watched papers and review ratings~\cite{hotcrp}.
%
HotCRP currently retains past conference data forever and requires manual
requests for account removal~\cite{hotcrp:privacy}.
%
We wrote two \xx specifications for HotCRP: conference anonymization to
protect old conference reviews, and GDPR account removal with
return.
%

%
\textbf{Conference anonymization} is invoked by PC chairs after the conference
and decorrelates users from their submissions, reviews, comments, and
per-user data such as watched papers.
%
User accounts remain in the database with no associated data.
%
Conference anonymization protects users' data after the conference; with \sys,
users can come back to view or edit their anonymized reviews and comments.
%

\textbf{Account removal} \one{} removes the user's account; \two{} removes
information only relevant to the user, such as their review preferences;
\three{} removes their author relationships to papers; and \four{} decorrelates
the remainder of their data, such as reviews. %, to individual pseudoprincipals.
%
Decorrelating a review removes its association with the reviewing user, but
importantly keeps the review itself around to preserve utility for others (\eg
the PC and the authors of the reviewed paper).
%makes them difficult to reassociate with one another or with the natural
%principal.
%
%Pseudoprincipals have suitable default properties provided by HotCRP; for
%example, they are marked as disabled, so that they have no permissions and
%cannot log in.
%
%This account removal \xxing transformation removes a user's relationship to
%co-authored papers, but does not remove the papers themselves.
%
%A different policy might go even further and remove papers whose final author
%is removed.
%
With \sys, users can remove their accounts even after conference anonymization has taken
place, and can always restore their accounts.% to the state prior to removal.
%

