%
\todo{table with components and loc}
\sys's prototype consists of $\approx10$k lines of Rust, and supports MySQL.
%(1.6k LoC specifically added for \syscrypt).
%
\sys also provides an API server that exposes a JSON-HTTP API that applications
can use to invoke \sys.
%
%In the following, we discuss key implementation details that \sys relies for
%security and performance.


%%%%%%%%%%%%%%%%%%%%%%%%%%%%%%%
%\section{API}
%\label{s:api}
%%
%An application can use the prototype if:
%\one{} it uses a MySQL database;
%\two{} rows to \xx have direct foreign key relationships to a users table,
%where each user corresponds to a row of that table;
%\three{} all rows to \xx are owned by one or more principals; and
%\four{} all rows can be uniquely identified (\eg via primary key).
%%
%Applications that do not satisfy these assumptions---\eg because they have
%complex ownership chains or use a NoSQL database---could be supported with
%extensions to the prototype.
%
\iffalse
\begin{figure}[t]
\begin{lstlisting}[style=rust,escapeinside={(*}{*)}]
// Generates keypair for p and returns
// the user's backup credential
(*\textbf{RegisterPrincipal}*)(p: UID, pw: Password,
        pubkey: PublicKey, privkey: PrivKey)
    -> Credential;
// \xxs principal p according to the spec,
// optionally over already-\xxed data ((*\S\ref{s:composition}*))
(*\textbf{\xxData}*)(p: Option<UID>, spec: \xxSpec,
    \xx_over: Option<(
        Credential,
        Vec<Locator>
    )>) -> (\xxID, Vec<Locator>);
// Reveals data \xxed by s for p with p's
// private key.
// locs point to bags p can decrypt
(*\textbf{RevealData}*)(p: Option<UID>, s: \xxID,
    cred: Credential, locs: Vec<Locator>);
// Gets principals at loc that p can speak-for.
(*\textbf{CanSpeakFor}*)(p: UID, cred: Credential,
    loc: Locator) -> Vec<UID>;
\end{lstlisting}
\caption{\sys's high-level API (Rust-like syntax).}
\label{f:api-high}
\end{figure}
%
\fi


%%%%%%%%%%%%%%%%%%%%%%%%%%%%%%%%%%%%%%%%%%%%%%%%%%%%%%%%%%%%%%%%%%%%%%%%%%%%%%%%%%%%%%%%%%
\section{Secure Record Storage}
%\sys stores the \xx and principal tables in the database.
%
When encrypting diff and speaks-for records, \sys appends a random nonce to
the record plaintext to prevent known-plaintext attacks.
%
It then generates a new public/private keypair for x25519 elliptic curve key
exchange.
%
Using the newly created private key and the principal's public key, \sys
performs the x25519 elliptic curve Diffie-Hellman ephemeral key exchange to
generate a shared secret.
%
\sys encrypts the record data with the shared secret, and saves the
ciphertext along with the freshly generated public key (required to decrypt
the data given the principal's private key).
%
This public key algorithm lacks key anonymity, so an attacker can determine
which records belong to the same principal, but this is not
fundamental~\cite{anonymous-keys}.

%Locators group together related diff and speaks-for records by principal and
%transformation; they are opaque in order to hide the principal's identity.
%
%Maintaining a locator set allows \sys's to easily find which records correspond
%to a principal's \xxed data to \eg reveal it (\S\ref{s:reveal}).

%%%%%%%%%%%%%%%%%%%%%%%%%%%%%%%%%%%%%%%%%%%%%%%%%%%%%%%%%%%%%%%%%%%%%%%%%%%%%%%%%%%%%%%%%%i
\section{Reveal Credentials}
%
\sys supports two forms of reveal credentials: \one{} private keys; or
\two{} principals' passwords, and recovery
tokens in case they forget their passwords.
%
If an application chooses to use the latter, it provides the principal's password
to \sys upon user registration.
%
\sys uses a variant of Shamir's Secret Sharing~\cite{secretsharing}
to generate three shares from the private key, any two of which can reconstruct the
private key.
%
Shares are $(x, f(x) \bmod p)$ tuples, where $f(x) = \fn{privkey} + \fn{rand} \cdot
x$ and $p > \fn{privkey}$ is a known prime.
%
One share derives $x$ from the user's password using a Password-Based Key
Derivation Function (PBKDF)~\cite{pbkdf-rfc}.
%
\sys stores the resulting $f(x)$ half of the share, allowing \sys to derive one
full share from the password.
%
%\sys generates the remaining two shares via randomly sampled $x$ coordinates.
%
\sys returns the second full share as a recovery token and stores the third full
share.
%
\sys can combine this third share with the recovery token
or a full share derived from the password to recover the
private key.
%if the user provides either
%their recovery token, or their password (from which \sys derives a full share).
%either derived from the password, or the recovery token.}
%
%if \sys obtains one additional share from either the password or from a recovery token,
%\sys can recover the private key.}

The PBKDF ensures that \sys cannot guess the password-derived value with
dictionary and rainbow table attacks~\cite{pbkdf}, and that \sys cannot brute force
the recovery token.
%
%\sys cannot reconstruct the private key without either the backup share or the
%other half of the password-derived share, and \sys cannot brute force the
%value of the backup share.
%

%
%\sys stores its shares---both its full share, and its half of the
%password-derived share---at an index equal to the hash of the user's ID and
%password in a separate shares table.
%%
%If the user loses their password and instead provides their recovery
%token, the user also provides this index so \sys can efficiently find the
%relevant shares.
%\sys thus returns both the backup share and the index as backup credentials in
%\fn{RegisterPrincipal}.
%
%\sys derives the private key given either form of credentials: given the user's
%password, \sys reconstructs the third share as well as the index; or, if the
%user forgets their password, they can provide the backup credentials instead.
%\sys combines the derived share with the one it stored to reconstruct the
%private key.

%When principals change their passwords or keypairs, applications re-register the
%principal with \sys.
%%
%To change a password, the principal provides their existing
%credentials (password or backup) and new password; \sys then derives
%their private key, redoes the secret sharing process described with the
%new password, and returns the new backup credential.
%%
%Users can similarly change their public/private keypair.
%%

%
Password-based secret-sharing is only one possible implementation for backup
secrets; \sys could also support password-based backup secrets by, for example, storing a
version of the private key encrypted with the user's password.
%

%%%%%%%%%%%%%%%%%%%%%%%%%%%%%%%%%%%%%%%%%%%%%%%%%%%%%%%%%%%%%%%%%%%%%%%%%%%%%%%%%%%%%%%%%%i
\section{Application Updates Replay Log}

\sys's prototype accepts application update specifications as Rust functions from database objects
($<$table, row$>$ pairs) to a new set of database objects. \sys stores the log
as a vector of updates function pointers with timestamps, and persists the
pointers to disk. This approach requires recompiling \sys to add new update
functions; an alternative approach could put the functions into separate
executables (and \sys would store executable names instead of function
pointers). In this alternative, \sys would invoke these executables upon reveal
with serialized input rows, and deserialize the output.

%%%%%%%%%%%%%%%%%%%%%%%%%%%%%%%%%%%%%%%%%%%%%%%%%%%%%%%%%%%%%%%%%%%%%%%%%%%%%%%%%%%%%%%%%%i
\section{Batching}
\label{s:impl:batching}

For efficiency, \sys batches the deletion of rows from the same table, updates
to rows of the same table from modifications and decorrelation, and the creation
of pseudoprincipals upon disguise. 
%
\sys also batches the deletion of pseudoprincipals during reveal, and 
the reveal of same-table diff records. This allows, for example, all updates for
database management operations to be applied to all diff record rows of the same
table at the same time; and for all rows of the same table to be inserted,
updated, or removed from the database in a single query.
%
%\sys performs decorrelation and modification operations individually for each
%row during disguise, as different rows are updated to different values.
%%
%\todo{xxx batch disguising?}

%%%%%%%%%%%%%%%%%%%%%%%%%%%%%%%%%%%%%%%%%%%%%%%%%%%%%%%%%%%%%%%%%%%%%%%%%%%%%%%%%%%%%%%%%%i
%
\section{Concurrency}
%
\sys runs \xxing and revealing transformations in transactions, providing
serializable isolation to application users.
%
If a query within a transformation fails, the entire transformation aborts
(returning an error to the application).
%
\sys provides an option to run long-running transformations that
touch large amounts of data (\eg anonymization of all users' posts) without a
transaction, at the expense of clients potentially observing intermediate
states.
%


%%%%%%%%%%%%%%%%%%%%%%%%%%%%%%%%%%%%%%%%%%%%%%%%%%%%%%%%%%%%%%%%%%%%%%%%%%%%%%%%%%%%%%%%%%i
%\head{Pre-generated Pseudoprincipal Keypairs.}
%%
%Generating a single private-public keypair is expensive ($\approx$300ms), but
%decorrelating disguises require creating many pseudoprincipals.
%%
%To make pseudoprincipal creation cheap, \sys maintains a pool of
%pregenerated keypairs for pseudoprincipals that it periodically refreshes.
%%
%The application tells \sys an appropriate size for the pool (\eg an upper bound
%on how much data a user might have).
%%
%If an attacker compromises the key pool, an admin must generate a new key pool and
%reset any public/private keypairs that were accessed since the compromise.
%%but \sys makes no security guarantees
%%about  the point when the compromise happens.
%
%%To recover, an admin must generate a new key pool and reset
%%extant principals' keys.
%

%Security against an attacker who compromises the server and observes future
%disguises is unaffected, as \sys only keys from the pool when it generates
%pseudoprincipals during disguises (when the keys would otherwise have been
%generated, and leaked to the attacker).
%%
%However, an attacker who compromises the server, exfiltrates the keys, and
%leaves would be able to come back and use the keys later to reveal disguised
%data.
%%
%A fixed-size pool and periodic refreshes bound the period of vulnerability and
%amount of data such an attacker can reveal.
%

