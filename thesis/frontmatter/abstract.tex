Many users today have tens to hundreds of accounts with web services that store
sensitive data, from social media to tax preparation and e-commerce
sites~\cite{tens,hundreds,password_life_cycle}.
%
While users have the right to delete their data (via \eg the
GDPR~\cite{eu:gdpr} or CCPA~\cite{ccpa}), users want and deserve more nuanced
controls over their data that don't exist today.
%
For example, a user might wish to hide
and protect data of an e-commerce or dating app profile when inactive, but also
want their data to be present should they return to use the application. 
%
Today, however, services often provide only coarse-grained, blunt tools that
result in all-or-nothing exposure of users’ private information.
%

%
This thesis introduces the notion of \emph{disguised data}, a reversible state
of data in which sensitive data is selectively hidden.
%
To demonstrate the feasibility of disguised data, this thesis also presents
\sys---the first system for disguised data---which helps database-backed web applications allow users
to remove their data without permanently losing their accounts, anonymize their
old data, and selectively dissociate personal data from public profiles.
%
\sys helps developers support these features while maintaining application
functionality and referential integrity via \emph{disguising} and
\emph{revealing} transformations.
%
Disguising selectively renders user data inaccessible via encryption, and
revealing enables the user to restore their data to the application.
%
\sys's techniques allow transformations to compose in any order, \eg deleting a
previously anonymized user's account, or restoring an account back to an
anonymized state.
%
With \sys, web applications can enable flexible privacy features with reasonable
developer effort (<1k LoC for a 160k real web application) and moderate
performance impact on application operation throughput (2--6\% in the common
case, varying with load). \todo{not sure what number to put for the throughput;
the expensive user is a little egregious.} 
%
Thus, this thesis suggests that privacy features beyond what applications offer
today are indeed practical, and can and should be widely supported.
%
