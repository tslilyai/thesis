Users' privacy requirements for their sensitive data on web applications are
highly contextual and constantly in flux. For example, a user might wish to hide
and protect data of an e-commerce or dating app profile when inactive, but also
want their data to be present should they return to use the application. 
%
Today, however, services often provide only coarse-grained, blunt tools that
result in all-or-nothing exposure of users’ private information.
%

%
This thesis introduces the notion of \emph{disguised data}, a reversible state
of data in which sensitive data is selectively hidden.
%
This thesis then describes \sys---the first system for disguised data---which
helps web applications allow users to remove their data without permanently
losing their accounts, anonymize their old data, and selectively dissociate
personal data from public profiles.
%
\sys helps developers support these features while maintaining application
functionality and referential integrity via \emph{disguising} and \emph{revealing}
transformations.
%
Disguising selectively renders user data inaccessible via encryption, and
revealing enables the user to restore their data to the application.
%
\sys's techniques allow transformations to compose in any order, \eg deleting a
previously anonymized user's account, or restoring an account back to an
anonymized state.
%