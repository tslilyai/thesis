\section{Takeaways from \sys}
%
\sys is a first step towards a world in which web services
routinely manage, store, and reveal \xxed user data.
%
\sys enables developers to provide data \xxing and revealing
transformations that give users control over their data in web applications.
%
These transformations help users protect inactive accounts, selectively dissociate
personal data from public profiles, and remove a web service's access to their
data without permanently losing their accounts.
%

\section{Extensions to \sys}
While \sys's design presents a basic system for disguising and revealing, \sys
itself has limitations that can be improved. 
%
As mentioned throughout, opportunities remain to optimize \sys's prototype
(\eg adding support for batching of queries) and implement useful features 
(\eg support for non-MySQL databases). 
%

One such area for exploration could investigate a plan for \sys to garbage
collect disguised data.  Until a user reveals their disguised data, \sys
currently retains the data in its disguise table, which could be forever if
users choose to never reveal data.
%
\sys could allow applications to put reasonable, coarse-grained time limits (\eg
10 years) on disguised data to eventually clean it up, without leaking
fine-grained information about which data was disguised at the same time.
%Given the amount of data that applications are willing to
%store and retain nearly indefinitely, we imagine that long timeouts will not be
%an issue.

%
Futhermore, services currently hold lots of disguised data, but cannot derive
value from it because it remains confidential. But perhaps with some initial
setup and some clever cryptography such as \todo{homomorphic encryption stufff},
Edna could allow some limited computation over disguised data.
%
Edna can allow computation over disguised data, such as aggregates (we
experimented with additive homomorphic encryption, and found it limited but
potentially promising) Needed to know computation groups in advance.

\section{Disguising Beyond \sys}
Zooming out beyond, we can contemplate completely different directions in which
disguising for flexible privacy could be applied. One example I’ll dive into
here is the direction of flexible access control, where the application itself
needs to control user data exposure to insiders.

Today, thousands of insiders at companies have access to sensitive user data,
and this access can lead to devastating data leaks.

One important part of this problem stems from the fact that today, employees
have access to all the data that they need for their job at any point in time.
For example, a hotel employee might be given authorization to see the full room
booking and customer database because they need this information to perform
their job, such as performing aggregates to determine room occupancy; or helping
customers manage their bookings

But they do not need all this information all the time!


We envision a world where every employee operates atop a personalized redacted
database that contains the minimum amount of data they need continuously. So in
this simplified example, when not actively helping a customer, a hotel customer
service representative might only need to be able to look at room occupancy and
see free rooms.

And a business intelligence analyst needs to be able to compute aggregates and
other statistics about where customers come from and where marketing should
focus, but not need to see user identifiers like usernames or precise phone
numbers.


But clearly these redactions are only the minimal amount of data the employee
needs for common, daily operations; when, for example, a particular customer
needs assistance, the employee currently helping the customer can change their
redactions temporarily and dynamically to expose only that user’s data (and even
then, perhaps only parts of it).

And another employee might simultaneously be helping another customer with
another booking, and apply a different upgrade to get a different partially
redacted view of the database.

And as soon as this data is no longer needed the employee’s access returns back
to its original, minimally exposing state.

To achieve this, we need a database system that can dynamically change
redactions that apply to each employee, but present a consistent view of the
redacted database to each employee.  And we can’t do this with unique database
views, because that would require an enormous number of views (one for each need
to upgrade data, and at a per-employee basis).

Redactions would need to handle referential integrity, work even in the presence
of indexes and database views, and work with existing query optimizations
without overly affecting performance.

Employees must also have the ability to flexibly perform fine-grained upgrades
of their access, without compromising security.

So clearly there’s a lot of exciting research here into solving the different
challenges posed by a redaction-based approach would require.


