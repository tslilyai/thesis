\sys is a first step towards a world in which web services
routinely manage, store, and reveal \xxed user data.
%
In this setting, new questions and directions for research arise.
%

\section{Retention of Disguised Data.}
%
When a user reveals their data, \sys removes it from the disguise table.
%
However, \sys currently retains disguised data until a user reveals it, which
could be forever if users choose to never reveal data.
%
\sys could allow applications to put reasonable,
coarse-grained time limits (\eg 10 years) on disguised data to eventually clean
it up, without leaking fine-grained information about which data was disguised
at the same time.
%Given the amount of data that applications are willing to
%store and retain nearly indefinitely, we imagine that long timeouts will not be
%an issue.

\section{Reveal Semantics.}
%
\sys today provides basic correctness guarantees when revealing data, but
further work might make \sys's current reveal semantics more precise.
%

Consider an example: a user's disguise modifies posts to scrub their username
from it, and a moderator later edits posts to remove swear words.
%
As the disguise modifies the post, \sys today does not restore the
original post content upon reveal, since the application has modified the post.
%
However, \sys could restore the post if it knew how to sequently remove the
swear words again.
%
Likewise, if \sys removed the post instead of scrubbing its content, the
moderator would never see it (and could not edit it), but \sys would reveal the
post with swear words still present.
%
In the first scenario, \sys knows that an update was applied, and refuses to
reveal the modified post; in the second, \sys does \emph{not} know that
moderation happened and reveals the removed post.
%
Neither might be what the application desires.
%

\sys could handle this situation by tracking operations applied in a replay log.
%
The application would invoke \sys when it performs operations that need to hold
over revealed data---\eg moderations or schema changes---to log these updates
(as \eg SQL queries) in \sys's replay log.
%
When revealing data, \sys would apply every relevant entry in the replay log
to the data about to be restored into the database.
%
This approach faces some limitations, such as assuming deterministic changes
and requiring additional application changes, and would need to ensure that
the replay log can be stored and applied efficiently.
%

\section{Pseudoprincipal references.}
%
\sys currently supports a global specification for checking and fixing references to
pseudoprincipals.
%
\sys could also support a menu of options, such
as per-table checks and fixes (where the developer to specifies per-table
policies) or per-inserted-object ones (where the developer makes application
modifications to log all added references to pseudoprincipals).

\section{Conclusion}
\label{s:concl}

%
\sys enables developers to provide data \xxing and revealing
transformations that give users control over their data in web applications.
%
These transformations help users protect inactive accounts, selectively dissociate
personal data from public profiles, and remove a web service's access to their
data without permanently losing their accounts.
%

%
We used \sys to add seven \xxing transformations to three web applications, and
found that the effort required was reasonable, that \sys's \xxing and
revealing operations are fast enough to be practical, and that they impose
little overhead on normal application operation.
%

%
\sys is open-source at \url{https://github.com/tslilyai/edna}.
%