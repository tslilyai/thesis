%
\sys is a first step towards a world in which web services routinely manage,
store, and reveal \xxed user data.
%
\sys enables developers to provide data \xxing and revealing transformations
that give users control over their data in web applications.
%
These transformations help users protect inactive accounts, selectively
dissociate personal data from public profiles, and remove a web service's access
to their data without permanently losing their accounts.
%
We showed that \sys supports disguised data in real-world applications with
reasonable performance. However, disguised data and flexible user privacy more
broadly present several opportunities that \sys leaves unsolved.
%

%%%%%%%%%%%%%%%%%%%%%%%%%%%%%%%%
%%%%%%%%%%%%%%%%%%%%%%%%%%%%%%%%
\section{\sys in the Real World}

While \sys enables disguised data in many web applications, supporting disguised
data in all web applications---particularly those at scale---raises several
open questions. 

%
First, what happens when disguised data becomes too much for applications
to store, if \eg millions of users disguise their data at a time?  
%
Even though storage is historically cheap, and applications today willingly
store and retain user data nearly indefinitely, eventually disguised data
(particularly since it cannot be sold or processed) may burden the application.
%
\sys currently retains disguised data in its disguised table until a user reveals
their disguised data, which could be forever if users choose to never reveal
data.
%
Instead, \sys could allow applications to put reasonable, coarse-grained time
limits (\eg 10 years) on disguised data to eventually clean it up, without
leaking fine-grained information about which data was disguised at the same
time.
%

%
Second, is there a way to better incentive applications to store disguised data?
As mentioned, even if applications \emph{do} store disguised data, the cost of
doing so may not pay off, since applications cannot derive value from it because
it remains confidential. 
%
From our preliminary explorations with clever cryptography such as additive
homomorphic secret sharing as done in Zeph~\cite{zeph}, \sys could potentially
support limited computation such as aggregations over disguised data from which
an application can derive value.
%
However, such cryptographic approaches require some initial setup to determine
whose data can be aggregated with whose (in order to split the secrets
appropriately) and who should be able to reveal the result (can individual users
also perform aggregates, or just the application?)
Futhermore, defining groups of users whose data can be combined requires more thought into what these groups may reveal about their constituents.
%
Homomorphic encryption techniques also come with performance overheads, but
perhaps such processing could be done offline.

\todo{Distributed \sys, parallel \sys, what else is needed to make \sys real?}

%%%%%%%%%%%%%%%%%%%%%%%%%%%%%%%%
\section{Disguising Beyond \sys: Disguised Database Views}
Zooming out beyond, we can contemplate completely different directions in which
disguising for flexible privacy could be applied. One example I’ll dive into
here is the direction of flexible access control, where the application itself
needs to control user data exposure to insiders.

Today, thousands of insiders at companies have access to sensitive user data,
and this access can lead to devastating data leaks.

One important part of this problem stems from the fact that today, employees
have access to all the data that they need for their job at any point in time.
For example, a hotel employee might be given authorization to see the full room
booking and customer database because they need this information to perform
their job, such as performing aggregates to determine room occupancy; or helping
customers manage their bookings

But they do not need all this information all the time!


We envision a world where every employee operates atop a personalized redacted
database that contains the minimum amount of data they need continuously. So in
this simplified example, when not actively helping a customer, a hotel customer
service representative might only need to be able to look at room occupancy and
see free rooms.

And a business intelligence analyst needs to be able to compute aggregates and
other statistics about where customers come from and where marketing should
focus, but not need to see user identifiers like usernames or precise phone
numbers.


But clearly these redactions are only the minimal amount of data the employee
needs for common, daily operations; when, for example, a particular customer
needs assistance, the employee currently helping the customer can change their
redactions temporarily and dynamically to expose only that user’s data (and even
then, perhaps only parts of it).

And another employee might simultaneously be helping another customer with
another booking, and apply a different upgrade to get a different partially
redacted view of the database.

And as soon as this data is no longer needed the employee’s access returns back
to its original, minimally exposing state.

To achieve this, we need a database system that can dynamically change
redactions that apply to each employee, but present a consistent view of the
redacted database to each employee.  And we can’t do this with unique database
views, because that would require an enormous number of views (one for each need
to upgrade data, and at a per-employee basis).

Redactions would need to handle referential integrity, work even in the presence
of indexes and database views, and work with existing query optimizations
without overly affecting performance.

Employees must also have the ability to flexibly perform fine-grained upgrades
of their access, without compromising security.

So clearly there’s a lot of exciting research here into solving the different
challenges posed by a redaction-based approach would require.


