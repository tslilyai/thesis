This thesis envisions a world in which web services routinely
protect, store, and reveal \xxed user data with user permission.
%
To move towards this goal, this thesis presents \sys, a system that gives users 
flexible privacy control over their data via data disguising.
%
With \sys, developers can provide data \xxing and revealing transformations that
help users protect inactive accounts, selectively dissociate personal data from
public profiles, and remove a web service's access to their data without
permanently losing their accounts.
%
This thesis demonstrates that \sys can support disguised data in real-world
applications with reasonable performance.
%

%
However, \sys is just the beginning. To truly achieve a world in which web
applications meet flexible privacy demands, several challenges remain unsolved,
which we describe here as future opportunities for research.
%

%%%%%%%%%%%%%%%%%%%%%%%%%%%%%%%%
\section{Deploying Disguised Data}
\label{s:deploying}

Supporting disguised data in web applications that have millions of users,
operate with low cost margins, or distribute their compute and storage
across multiple regions requires addressing several issues.

%
\paragraph{Disguised Data Storage.} 
With millions of users disguising their data, storage may become costly for
applications.
%
Even though storage is historically cheap, and applications today willingly
store and retain user data nearly indefinitely, disguised data may eventually
burden the application (particularly since it cannot be sold or processed).
%
\sys currently retains disguised data until a user
chooses to reveal it, which could be never.
%
To address storage concerns, \sys could allow applications to set reasonable,
coarse-grained time limits (\eg 10 years) on disguised data, after which \sys
will delete it.
%without
%leaking fine-grained information about which data was disguised at the same
%time.
%

%
\paragraph{Deriving Value from Disguised Data.} 
%
Support for disguised data enables users to protect their privacy and return to
the application if they wish---a potentially better solution than permanent user
deletion or decentralized approaches that move all user data out of the
application. However, with \sys, applications now store disguised data, but
cannot derive value from it, as it remains encrypted and confidential.

%While \sys enables an application to keep as much data as possible if users
%wish to
%protect their privacy via disguising and to return to the application,
%For many applications, this provides a better solution than handling user data
%deletion requests via permanent user deletion, or decentralized approaches that
%move all user data out of the application.

%
Cryptographic techniques such as additive homomorphic secret sharing as done in
Zeph~\cite{zeph}, may enable \sys to support limited computation over disguised
data (\eg aggregations) and allow applications to derive value from the
encrypted data.
%
However, such cryptographic approaches require some initial setup to determine
whose data can be aggregated with whose (in order to split the secrets
appropriately) and who should be able to reveal the result. Furthermore, defining
groups of users whose data can be combined raises concerns about what these
groups may reveal about their constituents.
%

\paragraph{\sys in Large-Scale Applications.}
Many web services, particularly those at scale, run on machines in multiple 
regions and datacenters, and replicate and shard their database.
%
While distributed databases used by these applications (\eg
Spanner~\cite{spanner}) already implement parallel, distributed, and large-scale
transactions, \sys's disguising and revealing transactions may potentially
impose high burdens on distributed databases due to the size and frequency of
these transactions.
%
For example, a disguise might access many database tables spread across multiple
datacenters.
%

%
Today, many applications already break down large transactions into smaller
pieces (at the cost of transactional consistency) or run large transactions in
batches overnight to reduce their impact on database performance.
%
Similarly, we can imagine \sys breaking down large disguise and reveal transactions
into multiple, per-table transactions, in order to not lock all tables at once; or
delaying their execution until night and notifying the user once their
disguise or reveal has completed. 
%

%%%%%%%%%%%%%%%%%%%%%%%%%%%%%%%%
\section{Disguising Beyond \sys: Flexible Access Control}
Disguising and revealing in \sys protects disguised data from parties external
to the owning user. However, disguising and revealing can also be useful in more
nuanced settings, where data is protected from different parties with different
disguises.
%a user may want different disguises to protect their data from different
%parties.  in more nuanced settings, such as disguising data in different ways
%to protect from different parties. 
%
In these settings, disguising and revealing may offer more flexible access
control to regulate exposure of user data to web application insiders.
%

%
In collaboration with Hannah Gross~\cite{funhouse}, we investigated using
disguising abstractions to allow applications to control exposure of user data
to insiders. Such an approach could help organizations whose employees
need to query some data, but may not need to see all of it.
%
Our work draws on the key observation that company employees do not
\emph{always} need all of the user information that they can access.
%
Just as \sys allows users to expose their data only while using the application,
perhaps the application can use disguising and revealing to only expose user
data to employees when they actually need and use it.
%

%
We envision a world where every employee uses a personalized disguised database
that contains the minimum amount of data they require to normally work. For
example, a hotel customer service representative might only need to be able to
look at room occupancy and find free rooms when not actively helping a customer.
%
%Similarly, a business intelligence analyst might need to be able to compute
%aggregates and statistics using coarse-grained data like area codes, but not
%need to see unique user identifiers like usernames or precise phone numbers.
%%
%
When an employee temporarily needs more data, their disguised database would
reveal only the minimum amount of data the employee needs. For example, if a
particular customer needs assistance, the employee currently helping the
customer should be able to temporarily reveal parts of their disguised database
to dynamically expose that user's data.
%
Different employees simultaneously helping different customers can reveal
different disguises to get a partially disguised view of the database
individualized for that employee.
%
As soon as the employee no longer needs the user's data, the employee’s access
returns back to its original, maximally disguised state.
%

%
To achieve this flexible access control, we need a database system that applies
disguises and dynamically reveals them at a per-employee granularity.
%
A strawman solution might attempt to do so with personalized database views
(similar to the proposal in Multiverse DB~\cite{multiverse}), but this would
require an enormous number of views (potentially one for each employee, and for
each reveal the employee requests).
%
Furthermore, just as with \sys, disguises cannot break the database: they would
need to handle referential integrity, work even in the presence of indexes and
database views, and work with existing query optimizations without overly
affecting performance.
%
Employees must also have the ability to flexibly perform fine-grained reveals of
their disguised views without compromising security.
%
While \sys's disguising and revealing abstractions can help address some of
these challenges (\eg handling referential integrity with decorrelation),
achieving a database system for flexible access control requires solving many
new challenges.

%
%Today, thousands of insiders at companies have access to sensitive user data,
%and this access can lead to devastating data leaks.
%
%One important part of this problem stems from the fact that today, employees
%have access to all the data that they need for their job at any point in time.
%For example, a hotel employee might be given authorization to see the full room
%booking and customer database because they need this information to perform
%their job, such as performing aggregates to determine room occupancy; or helping
%customers manage their bookings
%
%%%%%%%%%%%%%%%%%%%%%%%%%%%%%%%%
\section{Final Thoughts}
User data increasingly lands in the hands of web services, eroding users'
ability to control the privacy of their data as they wish.
%
While \sys represents just one step towards improved user data privacy on
the web, this thesis demonstrates that with disguised data,
applications can support flexible privacy features beyond those offered to users
today.
%
\sys provides evidence to both web services and regulators that flexible privacy
features can be supported with reasonable effort, and thus can and should be
widely implemented and enforced.
%
