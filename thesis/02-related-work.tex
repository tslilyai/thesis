\sys addresses the problem of disguised data for flexible data privacy in web
applications.
%\sys addresses the problem of reversible and composable data transformations
%for flexible data privacy in web applications.
%
Existing systems aim instead to support data deletion, prevent unauthorized data
access, or protect against database server compromise---valuable, but
complementary goals to support for disguised data.%\sys's.

\section{Compliance Tools}
Compliance tools such as K9db~\cite{k9db} and GDPRizer~\cite{gdprizer} help
correctly extract or delete data in response to compliance-related
requests.
%
These tools include systems to support wholesale user data extraction or deletion by tracking data
ownership by modifying the data layout~\cite{usershards, k9db}, tracking
information flow~\cite{schengendb}, or changing the
storage hardware~\cite{sdp}.
%

%
Tools like DELF at Meta~\cite{delf} focus on the related problem of data
deletion, letting developers specify deletion policies via
annotations on social graph edges and object types. DELF ensures correct
cascading data deletion, both for compliance-related data deletion requests and
for normal application deletions.
%

%
While one of the uses of disguising data is to provide GDPR-compliant account deletion, disguising
also supports more nuanced use cases beyond simple deletion: for example, \sys
allows users to return after deletion, hides old data for inactive users, or
hides some but not all data so the user can continue using the application.
%

\section{Policy Enforcement Systems}
Policy enforcement systems 
%such as Qapla~\cite{qapla} and Blockaid~\cite{blockaid} and
%others~\cite{static, jeeves, jif, hails, ifdb, oracle, multiverse, sieve}
aim to prevent unauthorized access to data and protect against leakage via
compromised accounts or SQL injections. 
%
These systems enforce developer-specified
visibility and access control policies via information flow
control~\cite{static, jeeves, jif, hails, ifdb}, authorized views~\cite{oracle},
per-user views~\cite{multiverse}, or by blocking or rewriting database
queries~\cite{blockaid, qapla, sieve}.
%
For example, a policy for a relational database-backed application may allow the
session user $U$ to access account information only if the predicate "\texttt{WHERE
accounts.user\_id = U}" returns true; this prevents an attacker who injects a
"\texttt{SELECT * FROM accounts}" query from reading any user accounts other than their
own.
%

%
Policy-enforcing systems do not help users anonymize data or maintain
application integrity constraints, which is the explicit goal with disguised
data.
%
By contrast, data disguising systems modify the database contents of sensitive
data---the data under disguise---so the data is no longer available in the
database, and thus unavailable even to the service itself. This is unlike policy
enforcement systems, which can deny access to sensitive data, but still
retains it in the database.
%

\section{Encrypted Storage Systems}
Encrypted storage systems such as CryptDB~\cite{cryptdb} and
Mylar~\cite{mylar} protect against database server compromise, with some
limitations~\cite{grubbs}.
%
These systems encrypt data in the database, and
ensure that only users with access to the right keys can decrypt the data.
%
Applications must handle keys, and send queries
either through trusted proxies that decrypt data~\cite{cryptdb}, or move
application functionality client-side~\cite{mylar}.
%
Encrypted databases have orthogonal goals to systems for disguised data: while
they protect data at all times against attackers who do not have the keys,
encrypted databases do not help applications protect against data access by
service itself, or by legitimate, authenticated users. 
%
Any user with legitimate access can view the data in an encrypted database,
whereas \xxed data is removed from the database and inaccessible to all users
and the application without permission from the data's owner.
%
%By constrast, systems for
%disguised data can anonymize or temporarily remove data from access by these 
%

\section{Other Related Work} 
Other work has also focused on protecting user data and giving users more
control over their data's exposure. However, these systems differ from \sys in
the specific problem they aim to solve, the threat model against which they
protect, and how they can be deployed with today's applications.

Some platforms focus on enforcing user-defined policies, instead of the
developer-defined policies that \sys supports. They prove that server-side
processing respects user-defined data policies via cryptographic
means~\cite{zeph} or systems security mechanisms~\cite{riverbed}.
%
However, this may restrict feasible application functionality (\eg to additively
homomorphic functions, as in Zeph~\cite{zeph}), or restrict combining data with
different policies, and requires modifying and redeploying existing
applications.
%
For example, Riverbed~\cite{riverbed} requires applications to run proxies on
both client and server, allowing the client proxy to attest that the server
indeed runs the correct proxy. 
%
Furthermore, users in Riverbed who define different policies cannot easily share
data with random other users at will, and aggregations over many users' data
become difficult.
%making certain applications less amenable to the Riverbed
%structure.
%\todo{Should I add more about Zeph deployment?}

%
Systems like Pesos~\cite{pesos}, Ironsafe~\cite{ironsafe}, and
Ryoan~\cite{ryoan} use trusted execution environments to protect all user data
and computation on this data against untrusted storage platforms.  Ryoan also
runs data processing applications as sandboxed modules within TEEs, thus
protecting user data against untrusted applications in addition to untrusted
platforms. 
By contrast, \sys protects only disguised data from the application and database
server, and trusts
developers to be
well-intentioned when writing disguise specifications.

%use trusted
%execution environments (TEEs) to enforce policy compliance in third-party
%storage platforms without trusting them with user data. 
%%
%Ryoan~\cite{ryoan} %
%By constrast, \sys does not rely on TEEs, and instead trusts developers to be
%well-intentioned.
%However, these systems require applications to be written with the systems'
%specific architecture in mind---for example, Ironsafe requires application to
%run atop Ironsafe's trusted policy compliance monitor; and in Ryoan, a developer
%needs to write their application as a computation DAG containing Ryoan
%instances, which each instance contains a data processing module.  \todo{more
%detail?}
%making these systems more difficult to deploy in
%today's applications.
%
%By contrast, \sys trusts developers to be well-intentioned, and \sys can support
%disguised data in applications without rearchitecting for TEEs.
%%

Decentralized platforms such as Solid~\cite{solid}, BSTORE~\cite{bstore},
Databox~\cite{databox}, and others~\cite{diy, amber, oort, w5, blockstack} put
data directly under user control, since users store their own data.
%
But decentralized platforms burden users with maintaining infrastructure, lack
the capacity for server-side compute, and break today's ad-based
business model.
%
By contrast, \sys leaves the data and business models unchanged,
and stores all data, including \xxed data, on the application's servers.
%

%
Devices using iOS~\cite{applesecurity}, Android~\cite{applesecurity}, or
CleanOS~\cite{cleanos} revoke data access via encryption, like \sys does.
%
However, these systems operate in settings that store only a single user's data;
disguised data applies in settings that include multiple users' data and shared
data.
%

%
Vanish~\cite{vanish} provides users with self-destructing data and a proof of
data deletion using decentralized infrastructure and cryptographic techniques
(with limitations against Sybil attacks~\cite{defeat_vanish}). Unlike \sys,
Vanish cannot restore deleted data and requires substantial application
restructuring to deploy (\eg the application must run a Vanish daemon, be aware of which
objects are ``vanishing objects,'' and store and access protected data
as key-value pairs instead of objects of other data structures).
%

%
Sypse~\cite{sypse} pseudonymizes user data and partitions personally identifying
information (PII) from other data. Instead of partitioning data, \sys modifies
the database and stores \xxed data encryptedly.
%

%
Finally, oblivious object stores like Dory~\cite{dory}, Snoopy~\cite{snoopy} and
Obladi~\cite{obladi} protect data and search access patterns against powerful
adversaries who can, for example, compromise the entire cloud software stack and
view metadata like network traffic and access patterns. To provide strong
security, these systems rely on complex encryption schemes, oblivious RAM,
hardware enclaves, or other cryptographic techniques.
%
However, oblivious object stores support only data retrival and simple queries
(\eg key/value point queries), and expect clients to perform most 
computations on data.
%
By contrast, \sys supports arbitrary server-side processing of undisguised
data, which most applications today require. 
%
Furthermore, Snoopy---the most recent system---achieves performance 39.1$\times$ slower than Redis, a
commonly-used non-oblivious object store, which is impractical for many 
applications.


%While these techniques have often been considered impractically slow,
%Snoopy---the most recent system---improves the throughput of data requests by
%running requests in batches.  However, for some applications, batched data
%requests may result in too high of an individual request latency.
%
%
%Many applications use relational databases in order to perform more
%complex queries, which \sys supports without any modification.
%

\iffalse
\begin{figure}[t]
    \centering
    \small
    \begin{tabular}{m{0.23\linewidth}|m{0.19\linewidth}|m{0.19\linewidth}|>{\RaggedRight\arraybackslash}m{0.19\linewidth}} %|m{0.08\linewidth}}
        \multirow{2}{*}{\centering\textbf{System}} &
            \multicolumn{3}{c}{\textbf{User $u$'s data is protected against...}}\\
        \cline{2-4}
            & \emph{SQL injection}
            & \emph{Compromised user $\neq u$}
            & \emph{Server compromise} \\
%            & \emph{Compromise of $u$} \\
        \hline
        Qapla~\cite{qapla} & \hfil \checkmark & & \\
        \hline
        CryptDB~\cite{cryptdb} & \hfil \checkmark & & \hfil \checkmark \\
        \hline
        \sys & \hfil \checkmark & \hfil \checkmark & \\
        \hline
        \syscrypt & \hfil \checkmark & \hfil \checkmark & \hfil \checkmark \\
    \end{tabular}
    \caption{Threats protected against by different classes of systems.}
    \label{tab:related_threats}
\end{figure}
\fi

