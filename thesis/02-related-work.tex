\sys aims to achieve disguised data for flexible data privacy in web applications.
%\sys addresses the problem of
%reversible and composable data transformations for flexible data privacy in web
%applications.
%
Existing systems aim instead to support data deletion, prevent unauthorized data
access, or protect against database server compromise---valuable, but
complementary goals to achieving disguised data.%\sys's.

\section{Data Deletion Tools}
Data deletion tools such as DELF at
Meta~\cite{delf} and K9DB~\cite{k9db} help correctly delete data.
%
DELF lets developers specify deletion policies via annotations on
social graph edges and object types, and ensures correct cascading data
deletion.
%
Other systems support wholesale user data deletion by tracking
data ownership by modifying the data layout~\cite{usershards, k9db} or
tracking information flow~\cite{schengendb}.
%
While disguising data can provide GDPR-compliant account deletion, disguising
also supports more nuanced use cases beyond simple deletion: for example, \sys
allows users to return after deletion, hides old data for inactive users, or
hides some but not all data so the user can continue using the application.
%
\sys could, however, benefit from these systems' proposed techniques to track a
user's data.
%

\section{Policy Enforcement Systems}
Policy enforcement systems 
%such as Qapla~\cite{qapla} and Blockaid~\cite{blockaid} and
%others~\cite{static, jeeves, jif, hails, ifdb, oracle, multiverse, sieve}
aim to prevent unauthorized access to data and protect against leakage via
compromised accounts or SQL injections.  They enforce developer-specified
visibility and access control policies via information flow
control~\cite{static, jeeves, jif, hails, ifdb}, authorized views~\cite{oracle},
per-user views~\cite{multiverse}, or by blocking or rewriting database
queries~\cite{blockaid, qapla, sieve}.

\lyt{Not sure if these systems belong here... but they are sort of
policy-enforcing?}
Other platforms can prove that server-side processing respects
user-defined instead of developer-specified data policies, via cryptographic means~\cite{zeph} or
systems security mechanisms~\cite{riverbed}.
%

%
Systems like Pesos~\cite{pesos} and Ironsafe~\cite{ironsafe} use TEEs to enforce policy compliance.
\lyt{TODO}.
%


This may restrict feasible application functionality (\eg to additively
homomorphic functions), or restrict combining data with different policies.
%

%
Policy-enforcing systems do not help users anonymize data or maintain
application integrity constraints, which is the explicit goal with disguised
data.
%
Database contents of sensitive data---the data under disguise---are also
modified so the data is no longer available in the database, and thus
unavailable even to the service itself. This is unlike policy enforcement
systems, which can deny access to sensitive data, but will still retain it in
the database.
%

\textbf{Encrypted Storage Systems}
Encrypted storage systems such as CryptDB~\cite{cryptdb} and
Mylar~\cite{mylar} protect against database server compromise, with some
limitations~\cite{grubbs}.
%
These systems encrypt data in the database, and
ensure that only users with access to the right keys can decrypt the data.
%
Applications must handle keys, and send queries
either through trusted proxies that decrypt data~\cite{cryptdb}, or move
application functionality client-side~\cite{mylar}.
%
Encrypted databases have orthogonal goals to systems for disguised data: while they protect data at all times
against attackers who do not have the keys, encrypted databases do not help applications
anonymize or temporarily remove data, which systems for disguised data do.
%
Any user with legitimate access can view the data in an encrypted database,
whereas \xxed data is removed from the database.

\section{Other Related Work} 
\subsection{Decentralized Platforms} 
Decentralized platforms such as Solid~\cite{solid}, BSTORE~\cite{bstore},
Databox~\cite{databox}, and others~\cite{diy, amber, oort, w5, blockstack} put
data directly under user control, since users store their own data.
%
But decentralized platforms burden users with maintaining infrastructure, lack
the capacity for server-side compute, and break today's ad-based
business model.
%
By contrast, \sys leaves the data and business models unchanged,
and stores all data, including \xxed data, on the application's servers.
%
\lyt{TODO: bluesky, other decentralized social media comapnies? Or stick with
papers?}

%
\subsection{Single-Device Data Encryption}
Devices using iOS~\cite{applesecurity},
Android~\cite{applesecurity}, or CleanOS~\cite{cleanos} revoke
data access via encryption, like \sys does.
%
However, these systems operate in settings that store only a single user's data;
disguised data applies in settings that include multiple users' data and shared
data.
%

%
\subsection{New Data Paradigms}
\lyt{TODO I'm not sure how to organize these...}

Vanish~\cite{vanish} provides users with self-destructing data and a proof of
data deletion using decentralized infrastructure and cryptographic techniques
(with limitations against Sybil attacks~\cite{defeat_vanish}). Unlike
\sys, Vanish cannot restore deleted data and requires extensive application
restructuring.
%

%
Sypse~\cite{sypse} pseudonymizes user data and partitions
personally identifying information (PII) from other data. Instead of
partitioning data, \sys modifies the database and stores \xxed data
encrypted.
%

%
SchengenDB~\cite{schengendb} proposes a new data design that uses fine-grained information flow
tracking to track the ownership and provenance of database
rows, and provide information to support user data deletion.
%

%
Oblivious object stores like Dory and Snoopy protect data and search access
patterns. \lyt{TODO}
Dory~\cite{dory}
Snoopy~\cite{snoopy}
%

\iffalse
\begin{figure}[t]
    \centering
    \small
    \begin{tabular}{m{0.23\linewidth}|m{0.19\linewidth}|m{0.19\linewidth}|>{\RaggedRight\arraybackslash}m{0.19\linewidth}} %|m{0.08\linewidth}}
        \multirow{2}{*}{\centering\textbf{System}} &
            \multicolumn{3}{c}{\textbf{User $u$'s data is protected against...}}\\
        \cline{2-4}
            & \emph{SQL injection}
            & \emph{Compromised user $\neq u$}
            & \emph{Server compromise} \\
%            & \emph{Compromise of $u$} \\
        \hline
        Qapla~\cite{qapla} & \hfil \checkmark & & \\
        \hline
        CryptDB~\cite{cryptdb} & \hfil \checkmark & & \hfil \checkmark \\
        \hline
        \sys & \hfil \checkmark & \hfil \checkmark & \\
        \hline
        \syscrypt & \hfil \checkmark & \hfil \checkmark & \hfil \checkmark \\
    \end{tabular}
    \caption{Threats protected against by different classes of systems.}
    \label{tab:related_threats}
\end{figure}
\fi

