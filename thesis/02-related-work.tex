\iffalse
\begin{figure}[t]
    \centering
    \small
    \begin{tabular}{m{0.23\linewidth}|m{0.19\linewidth}|m{0.19\linewidth}|>{\RaggedRight\arraybackslash}m{0.19\linewidth}} %|m{0.08\linewidth}}
        \multirow{2}{*}{\centering\textbf{System}} &
            \multicolumn{3}{c}{\textbf{User $u$'s data is protected against...}}\\
        \cline{2-4}
            & \emph{SQL injection}
            & \emph{Compromised user $\neq u$}
            & \emph{Server compromise} \\
%            & \emph{Compromise of $u$} \\
        \hline
        Qapla~\cite{qapla} & \hfil \checkmark & & \\
        \hline
        CryptDB~\cite{cryptdb} & \hfil \checkmark & & \hfil \checkmark \\
        \hline
        \sys & \hfil \checkmark & \hfil \checkmark & \\
        \hline
        \syscrypt & \hfil \checkmark & \hfil \checkmark & \hfil \checkmark \\
    \end{tabular}
    \caption{Threats protected against by different classes of systems.}
    \label{tab:related_threats}
\end{figure}
\fi

Disguised data provides flexible (and reversible) degrees of data privacy.
%\sys addresses the problem of
%reversible and composable data transformations for flexible data privacy in web
%applications.
%
Existing systems aim instead to support data deletion, prevent unauthorized data
access, or protect against database server compromise---valuable, but
complementary goals to achieving disguised data.%\sys's.

\textbf{Data deletion tools}, such as DELF at
Meta~\cite{delf} and K9DB~\cite{k9db} help correctly delete data.
%
%DELF lets developers specify deletion policies via annotations on
%social graph edges and object types, and ensures correct cascading data
%deletion.
%
%Other systems support wholesale user data deletion by tracking
%data ownership by modifying the data layout~\cite{usershards, k9db} or
%tracking information flow~\cite{schengendb}.
%
While disguising data can provide GDPR-compliant account deletion, disguising
also supports more nuanced use cases beyond simple deletion: for example, \sys
allows users to return after deletion, hides old data for inactive users, or
hides some but not all data so the user can continue using the application.
%
\sys could, however, benefit from these systems' proposed techniques to track a
user's data.
%

\textbf{Policy enforcement systems} such as Qapla~\cite{qapla},
Blockaid~\cite{blockaid} and others~\cite{static, jeeves, jif, hails, ifdb,
oracle, multiverse, sieve} aim to prevent unauthorized access to data and
protect against leakage via compromised accounts or SQL injections. 
%They enforce developer-specified
%visibility and access control policies via information flow
%control~\cite{static, jeeves, jif, hails, ifdb}, authorized views~\cite{oracle},
%per-user views~\cite{multiverse}, or by blocking or rewriting database
%queries~\cite{blockaid, qapla, sieve}.
%
However, policy-enforcing systems do not help users anonymize data or maintain
application integrity constraints, which is the explicit goal with disguised
data.
%
Database contents of sensitive data---the data under disguise---are also
modified so the data is no longer available in the database, and thus
unavailable even to the service itself. This is unlike policy enforcement
systems, which can deny access to sensitive data, but will still retain it in
the database.
%

\textbf{Encrypted storage systems} such as CryptDB~\cite{cryptdb} and
Mylar~\cite{mylar} protect against database server compromise, with some
limitations~\cite{grubbs}.
%
%These systems encrypt data in the database, and
%ensure that only users with access to the right keys can decrypt the data.
%
%Applications must handle keys, and send queries
%either through trusted proxies that decrypt data~\cite{cryptdb}, or move
%application functionality client-side~\cite{mylar}.
%
Encrypted databases have orthogonal goals to systems for disguised data: while they protect data at all times
against attackers who do not have the keys, encrypted databases do not help applications
anonymize or temporarily remove data, which systems for disguised data do.
%
Any user with legitimate access can view the data in an encrypted database,
whereas \xxed data is removed from the database.

\textbf{Other related work} uses crytographic or systems security methods to
prove enforcement of user-defined policies~\cite{zeph, riverbed}, or explores
new decentralized paradigms for web application operation~\cite{solid, bstore,
databox, diy, amber, oort, w5, blockstack}.  These may restrict application
functionality, whereas a system for disguised data can leave the data and business models unchanged.
% TODO vanish, single-device mobile data encryption
%
%%
%{Devices} using iOS~\cite{applesecurity},
%Android~\cite{applesecurity}, or CleanOS~\cite{cleanos} revoke
%data access via encryption, like \sys does.
%%
%However, these systems operate in settings that store only a single user's data;
%\sys instead tackles the problem of
%transformations that operate over multiple users' data and shared data without
%breaking the application.
%%
%
%%
%Vanish~\cite{vanish} provides users with self-destructing data and a {proof of
%data deletion} using decentralized infrastructure and cryptographic techniques
%(with limitations against Sybil attacks~\cite{defeat_vanish}). Unlike
%\sys, Vanish cannot restore deleted data and requires extensive application
%restructuring.
%%
%
%%
%{Sypse~\cite{sypse}} pseudonymizes user data and partitions
%personally identifying information (PII) from other data. Instead of
%partitioning data, \sys modifies the database and stores \xxed data
%encrypted.
%%
%
%{Decentralized platforms} such as Solid~\cite{solid}, BSTORE~\cite{bstore},
%Databox~\cite{databox}, and others~\cite{diy, amber, oort, w5, blockstack} put
%data directly under user control, since users store their own data.
%%
%But decentralized platforms burden users with maintaining infrastructure, lack
%the capacity for server-side compute, and break today's ad-based
%business model.
%%
%By contrast, \sys leaves the data and business models unchanged,
%and stores all data, including \xxed data, on the application's servers.
%%
%
%Some platforms can prove that server-side processing respects
%{user-defined policies} via cryptographic means~\cite{zeph} or
%systems security mechanisms~\cite{riverbed}. This may restrict feasible
%application functionality (\eg to additively homomorphic functions), or restrict
%combining data with different policies. \sys protects data only after
%\xxing, but allows unrestricted application functionality before
%\xxing.

\iffalse
%-------------------------------------------------------------------------------
\section{Related Work}
%-------------------------------------------------------------------------------
\label{sec:related}
\let\rwhighlight\relax

\sys is the first system to address the problem of reversible and composable data
transformations for selective data removal in web applications.
%
Existing systems aim instead to support data deletion, prevent
unauthorized data access, or protect against database server
compromise---valuable, but complementary goals to \sys's.
%
%\S\ref{s:eval-related} delves further into these system differences:
%we contrast \sys with Qapla (a policy enforcement system),
%and demonstrate how \sys can leverage encrypted storage
%to add protections for un\xxed data.

\textbf{Data deletion tools}, such as DELF at
Meta~\cite{delf}, help correctly delete data.
%
DELF lets developers specify deletion policies via annotations on
social graph edges and object types, and ensures correct cascading data
deletion.
%
Other systems support wholesale user data deletion by tracking
data ownership by modifying the data layout~\cite{usershards, k9db} or
tracking information flow~\cite{schengendb}.
%
While \sys also supports GDPR account deletion, \sys's focus is on more nuanced
use cases beyond simple deletion: \sys allows users to return after deletion,
hides old data for inactive users, or hides some but not all data so the user
can continue using the application.
%
\sys could, however, benefit from these systems' proposed techniques to track a
user's data.
%

\textbf{Policy enforcement systems} such as Qapla~\cite{qapla} aim to prevent
unauthorized access to data and protect against leakage via compromised
accounts or SQL injections. They enforce developer-specified
visibility and access control policies via information flow
control~\cite{static, jeeves, jif, hails, ifdb}, authorized views~\cite{oracle},
per-user views~\cite{multiverse}, or by blocking or rewriting database
queries~\cite{blockaid, qapla, sieve}.
%
Policy-enforcing systems do not help users anonymize data or maintain
application integrity constraints, which is \sys's explicit
goal.
%
Instead of denying access to data, \sys changes the database contents so sensitive
data is no longer available in the database, and thus unavailable even to the
service itself.
%

\textbf{Encrypted storage systems} such as CryptDB~\cite{cryptdb} and
Mylar~\cite{mylar} protect against database server compromise, with some
limitations~\cite{grubbs}.
%
These systems encrypt data in the database, and
ensure that only users with access to the right keys can decrypt the data.
%
Applications must handle keys, and send queries
either through trusted proxies that decrypt data~\cite{cryptdb}, or move
application functionality client-side~\cite{mylar}.
%
Encrypted databases have orthogonal goals to \sys's: while they protect data at all times
against attackers who do not have the keys, encrypted databases do not help applications
anonymize or temporarily remove data, which \sys does.
%
Any user with legitimate access can view the data in an encrypted database,
whereas \sys removes \xxed data from the database.
%changes the database contents to hide sensitive data.
%from anyone other than the owning user.

%Policy-enforcing and encrypted storage systems do not help users anonymize data
%or maintain application integrity constraints via anonymized data, which is
%\sys's explicit goal. Unlike these systems, \sys changes the database contents,
%so sensitive data is no longer available in the database (and thus unavailable
%even to the service itself).

%\lyt{Changed above.}
%
%Figure 1 shows
%the threats that policy enforcement systems (\eg Qapla~\cite{qapla}), encrypted
%databases (\eg CryptDB~\cite{cryptdb}, and \sys defend against.

\textbf{Other related work}.
%
% Other systems share \sys's goal of giving users more control over user
% data.
%
{Devices} using iOS~\cite{applesecurity},
Android~\cite{applesecurity}, or CleanOS~\cite{cleanos} revoke
data access via encryption, like \sys does.
%
However, these systems operate in settings that store only a single user's data;
\sys instead tackles the problem of
transformations that operate over multiple users' data and shared data without
breaking the application.
%

%
Vanish~\cite{vanish} provides users with self-destructing data and a {proof of
data deletion} using decentralized infrastructure and cryptographic techniques
\newstuff{(with limitations against Sybil attacks~\cite{defeat_vanish})}. Unlike
\sys, Vanish cannot restore deleted data and requires extensive application
restructuring.
%

%
{Sypse~\cite{sypse}} pseudonymizes user data and partitions
personally identifying information (PII) from other data. Instead of
partitioning data, \sys modifies the database and stores \xxed data
encrypted.
%

{Decentralized platforms} such as Solid~\cite{solid}, BSTORE~\cite{bstore},
Databox~\cite{databox}, and others~\cite{diy, amber, oort, w5, blockstack} put
data directly under user control, since users store their own data.
%
But decentralized platforms burden users with maintaining infrastructure, lack
the capacity for server-side compute, and break today's ad-based
business model.
%
By contrast, \sys leaves the data and business models unchanged,
and stores all data, including \xxed data, on the application's servers.
%

Some platforms can prove that server-side processing respects
{user-defined policies} via cryptographic means~\cite{zeph} or
systems security mechanisms~\cite{riverbed}. This may restrict feasible
application functionality (\eg to additively homomorphic functions), or restrict
combining data with different policies. \sys protects data only after
\xxing, but allows unrestricted application functionality before
\xxing.

\iffalse
\begin{figure}[t]
    \centering
    \small
    \begin{tabular}{m{0.24\linewidth}|m{0.18\linewidth}|m{0.19\linewidth}|>{\RaggedRight\arraybackslash}m{0.19\linewidth}} %|m{0.08\linewidth}}
        \multirow{2}{*}{\centering\textbf{System}} &
            \multicolumn{3}{c}{\textbf{User $u$'s data is protected against...}}\\
        \cline{2-4}
            & \emph{SQL injection}
            & \emph{Compromised user $\neq u$}
            & \emph{Server compromise} \\
%            & \emph{Compromise of $u$} \\
        \hline
        Qapla~\cite{qapla} & \hfil \checkmark & & \\
        \hline
        CryptDB~\cite{cryptdb} & \hfil \checkmark & & \hfil \checkmark \\
        \hline
        \sys & \hfil \checkmark & \hfil \checkmark & \\
        \hline
        \syscrypt & \hfil \checkmark & \hfil \checkmark & \hfil \checkmark \\
    \end{tabular}
    \caption{Threats protected against by different classes of systems. \sys
    also keeps the application functional if it relies on data
    that users want to protect.}
    \label{tab:related_threats}
\end{figure}
\fi

\iffalse

%\begin{figure*}[t!]
%    \centering
%    \small
%    \begin{tabular}{m{0.09\linewidth}|m{0.48\linewidth}||m{0.07\linewidth}|m{0.13\linewidth}|>{\RaggedRight\arraybackslash}m{0.1\linewidth}} %|m{0.08\linewidth}}
%        \multirow{2}{*}{\centering\textbf{System}} &
%            %\multirow{2}{*}{\parbox[m]{\linewidth}{\centering\textbf{Protection Mechanism}}} &
%            \multirow{2}{*}{\parbox[m]{\linewidth}{\centering\textbf{Protection Mechanism for user $u$'s data}}} &
%            \multicolumn{3}{c}{\textbf{User $u$'s data is protected against...}}\\
%        \cline{3-5}
%            & & \emph{SQL Injection}
%            & \emph{Compromised Authorized User $\neq u$}
%            & \emph{Root DB Server Access} \\
%%            & \emph{Compromise of $u$} \\
%        \hline
%        Qapla~\cite{qapla} & Rewrites SQL queries to prevent unauthorized access
%        to $u$'s data. & \hfil \checkmark & & \\
%        \hline
%        CryptDB~\cite{cryptdb} & Encrypts data with authorized users' credentials. & \hfil \checkmark & & \hfil \checkmark \\
%        \hline
%        \sys & \Xxs data by encrypting with $u$'s credentials \& removing from DB. &
%        \hfil \checkmark & \hfil \checkmark & \\
%        \hline
%        \syscrypt & \Xxs data by encrypting with $u$'s credentials \& removing from DB; encrypts
%        un\xxed data with authorized users' credentials. &
%        \hfil \checkmark & \hfil \checkmark & \hfil \checkmark \\
%    \end{tabular}
%    \caption{\sys protects user $u$'s \xxed data against
%    attackers who compromise authorized users---users with legitimate access to
%    $u$'s data (\eg an admin)---as well as application SQL injections,
%    since \sys encrypts \xxed data with only $u$'s key and removes it from the
%    database. Query rewriting and encrypted database systems
%    prevent SQL injection but allow a compromised authorized user to access $u$'s
%    sensitive data. Encrypted databases also address server compromise;
%    \syscrypt combines them with \sys.}
%
%    \label{tab:related_threats}
%\end{figure*}

%
\sys gives developers tools to remove and redact user data from a web
application without breaking the application.
%
Existing systems for data protection complement \sys, as they aim instead to
support data deletion, prevent unauthorized data access, or protect against
database server compromise.
%

%
Laws and regulations such as the EU's General Data Protection Regulation
(GDPR)~\cite{eu:gdpr}, the California Consumer Privacy Act (CCPA)~\cite{ccpa},
and others~\cite{va:privacy-act, china:gdpr-like} give users the right to
request deletion of their data from web applications.
%
\textbf{Data deletion tools} developed at large organizations, such as DELF at
Meta~\cite{delf}, help to correctly delete data in response to such requests.
%
DELF helps developers specify correct deletion policies via annotations on
social graph edges and object types, and ensures correct cascading data
deletion.
%

%
Like data deletion systems, \sys supports GDPR account deletion.
However, \sys also supports use cases beyond simple deletion.
%
\sys can \xx data in a GDPR-compliant manner and allow the user to return, \xx
old data for inactive users, or \xx some but not all data, allowing the user to
continue using the application.
%
%Finally, \sys emphasizes maintaining application functionality, which
%involves storing placeholder data and fake users in the database.
%

%
\textbf{Policy enforcement systems} such as Qapla~\cite{qapla} aim to prevent
unauthorized access to data and protect against leakage via compromised,
unauthorized accounts or SQL injections.
%
They enforce developer-specified visibility and access control policies via
information flow control~\cite{static, jeeves, jif, hails, ifdb}, authorized
views~\cite{oracle} or per-user views~\cite{multiverse}, or blocking or
rewriting database queries~\cite{blockaid, qapla, sieve}.
%
%These systems restrict access to user data by restricting a client's queries
%depending on the access policy of the client's application role.
%
%Users can only successfully query data if authorized by their application
%role's policy.% to view certain data can successfully query it.
%
%This addresses the threat of a compromised \emph{unauthorized} user account, as
%well as arbitrary SQL queries via SQL injections: in either case, the exposed
%data is restricted to what the policy allows.
%
%However, if a compromised user (\eg an admin) has a legitimate reason to view
%data and their policy authorizes broad access, policy enforcement will fail
%to contain the damage.
%

\textbf{Encrypted storage systems} such as CryptDB~\cite{cryptdb} and
Mylar~\cite{mylar} aim to protect against database server compromise,
with some limitations~\cite{grubbs}.
%
These systems encrypt data in the database, and ensure that only
users with legitimate reasons to view the data (\ie the owner or an
admin) can decrypt the data.
%
Applications handle keys, and send queries either through
trusted proxies that decrypt data~\cite{cryptdb}, or move application
functionality client-side~\cite{mylar}.
%
%However, they fail to prevent \emph{authorized} accesses from a compromised
%account that (for legitimate purposes) holds the key material to decrypt the
%data.

Policy-enforcing and encrypted storage systems do not help users anonymize data
or maintain application functionality in the face of anonymized data, which is
\sys's explicit goal.
%
Unlike these systems, \sys changes the database contents so sensitive data is no
longer available in the database.
%
\sys encrypts users' \xxed data, and ensures that the key material is
unavailable to the application and any privileged account.
%
Because they provide complimentary guarantees to \sys, \sys can leverage
policy-enforcement or encrypted storage to strengthen its protections for
un\xxed data; we explore this with \syscrypt \newstuff{(\S\ref{s:eval-cryptdb})}.
%
Figure~\ref{tab:related_threats} shows the threats that policy-enforcement
systems (\eg Qapla~\cite{qapla}), encrypted databases (\eg
CryptDB~\cite{cryptdb}), and \sys defend against.
%

% Adding an encrypted database to \sys also protects un\xxed data against full
% database server compromise.
% %(via encrypted storage).
% %
% This necessarily comes with the overheads associated with encrypted databases.
% %
% \S\ref{s:design-cryptdb} describes \syscrypt, an extension to \sys that
% leverages CryptDB-like encrypted storage; \S\ref{s:eval-cryptdb} evaluates the
% resulting costs.
%

%%%%%%%%%%%%%%%%%%%%%%%%%%%%%%%%%%%%%%%%%%%


\textbf{Other Related Work.}
%
Some systems use fine-grained information flow tracking~\cite{schengendb} to
track the ownership and provenance of database rows, and provide information to
support user data deletion.
%
\sys requires more developer input, but supports nuanced transformations like
decorrelation or partial anonymization of shared data.
%

%
Sypse~\cite{sypse} pseudonymizes user data and partitions personally-identifying
information (PII) from other data.
%
Instead of partitioning data, \sys actually modifies the data and stores \xxed
data encryptedly.
\lyt{We also had a todo about this, not sure if we want to add more...}
%

%
Decentralized platforms such as Solid~\cite{solid}, BSTORE~\cite{bstore},
Databox~\cite{databox}, and others~\cite{diy, amber, oort, w5, blockstack} put
data fully and directly under user control, since users store their own data.
%
But this burdens users with maintaining infrastructure, and decentralized platforms
lack the capacity for server-side compute and break today's ad-based
business model.
%
By contrast, \sys leaves the data and business models unchanged,
storing all data, including \xxed data, on the application's servers.
%

%
Some platforms can prove that server-side processing respects
\rwhighlight{user-defined data policies} via cryptographic means~\cite{zeph} or
systems security mechanisms~\cite{riverbed}.
%
This may restrict feasible application functionality (\eg to additively
homomorphic functions), or restrict combining data with different policies.
%
\sys only protects user data after \xxing, but in exchange allows unrestricted
application functionality before \xxing.
%

\fi
\fi
