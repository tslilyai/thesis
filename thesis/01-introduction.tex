%Users' privacy requirements for their sensitive data on web applications are
%highly contextual and constantly in flux. 
Users have tens to hundreds of accounts with web services that store
sensitive data, from social media to tax preparation and e-commerce
sites~\cite{tens,hundreds,password_life_cycle}.
%
%
While users have the right to delete their data (via \eg the
GDPR~\cite{eu:gdpr} or CCPA~\cite{ccpa}), more nuanced
data controls often don't exist.
%
For example, a user might wish to hide and protect their profiles on an
e-commerce or dating app when inactive, and to recover their accounts should
they return to use the application. 
%
However, services often provide only coarse-grained, blunt tools such as
permanent account deletion that
result in all-or-nothing exposure of users’ private data
%

%
This thesis introduces the notion of \emph{disguised data} for flexible data
privacy. Disguised data is a reversible state of data in which sensitive data is
selectively hidden.
%
To demonstrate the feasibility of disguised data, this thesis also presents
\sys---the first system for disguised data---which helps database-backed web applications allow users
to remove their data without permanently losing their accounts, anonymize their
old data, and selectively dissociate personal data from public profiles.
%
\sys lets developers support these features while maintaining application
functionality and referential integrity via \emph{disguising} and \emph{revealing}
transformations.
%
Disguising selectively renders user data inaccessible via encryption, and
revealing enables the user to restore their data to the application.
%
%\sys's techniques allow transformations to compose in any order, \eg deleting a
%previously anonymized user's account, or restoring an account back to an
%anonymized state.
%

%
This chapter motivates the need for data disguising for flexible privacy;
describes the challenges in disguising and revealing data; introduces our
approach to achieve data disguising; and summarizes the contributions of this
thesis.

%*********************************************************************************%
%*********************************************************************************%
%*********************************************************************************%
\section{Motivation} 
%Many users today have tens to hundreds of accounts with web services that store
%sensitive data, from social media to tax preparation and e-commerce
%sites~\cite{tens,hundreds,password_life_cycle}.
%%
%And while users now have the right to delete their data (via \eg the
%GDPR~\cite{eu:gdpr} or CCPA~\cite{ccpa}), users want and deserve more nuanced
%controls over their data that don't exist today.
%%
User today lack fine-grained controls over their personal data stored by web
applications.
%
Consider Twitter: after a change in management~\cite{musk-twitter}, many users
wanted to leave the platform and try out alternatives (\eg
Mastodon~\cite{mastadon}).  But each user faced a tricky question: should they
keep their Twitter account, or should they delete it?
%
Advice on how to quit Twitter~\cite{quit-twitter-india, quit-twitter-mash}
highlight how keeping an inactive account leaves sensitive information (\eg
private messages) vulnerable on Twitter's servers; but deleting the account
prevents the user from changing their mind and coming back, causing them to
permanently lose all their followers and content.  Hence, many users left
Twitter but kept their accounts~\cite{nbc-twitter,shondarhimes,kenolin}.
%
A better solution would let users temporarily revoke Twitter's access to their
data while having the option to come back.

Similarly, users give dating apps personal data, and frequently deactivate and
reactivate their accounts.
%
This sensitive data should be protected from the application and potential data
breaches~\cite{tinder, okcupid} when a user deactivates their account, but be
readily available when they choose to return.

Users may also prefer old data, such as past purchases in an online store or
their passport details with a hotel, to be inaccessible to the service after
some time of inactivity, and therefore protected from leaks or service
compromises~\cite{retention,breach:marriott}. Or users may prefer
to---explicitly or automatically---dissociate their identity from old data, such
as teenage social media posts or old reviews on HotCRP~\cite{hotcrp}.
%
Today, users work around the lack of such support by explicitly maintaining
multiple identities (\eg Reddit throwaway accounts~\cite{reddit:throwaway} and
Instagram ``finstas''~\cite{nytimes:finsta}), an inflexible and laborious
solution.
%

%
Providing this fine-grained privacy functionality can benefit both the service and the user.
%
It helps the service comply with privacy regulations, reduces its liability on
data breaches, and appeals to privacy-conscious users; meanwhile, the user can
rest assured that their privacy is protected, but can also get their data back
and reveal their association with it if they want.


%*********************************************************************************%
%*********************************************************************************%
%*********************************************************************************%
\section{Disguised Data: A New Abstraction for Flexible Privacy}
%
This desired flexible privacy functionality describes an in-between state of
users data in web applications. The data is not quite gone, because a user
can return to the application and restore their data; but the user data is
also not quite there, because some or all of it has been removed or made
inaccessible to the application.
%

%
To capture this new state, this thesis introduces the new 
abstraction of \emph{disguised data}.  Disguised data represents a state of data
where \one{} some or all of the user's original sensitive data is rendered
inaccessible to the application; \two{} some data may be replaced with
placeholders to keep the application structure intact (\eg placeholder parent
comments to maintain comment thread structure); and \three{} the data can be
restored with the user's authorization.
%

%
Systems for disguised data move closer to an Internet where
users can leave services and return at any time, where old data on servers is
protected by default, and where services provide users with control over their
identifying data visible to the service and other users.
%

%*********************************************************************************%
%*********************************************************************************%
%*********************************************************************************%
\section{Challenges} 
Today, disguised data for flexible privacy remains out of reach for users of
web applications in part because getting it right is hard for application
developers. In particular, developers face four main challenges.

%
First, real applications store their data in databases with complex schemas and have
complex notions of privacy, data ownership, and data sharing.
%
Simple solutions that might try to delete all data associated with a user can
create orphaned data or break referential integrity, which is the invariant that if a
table row references another row via a foreign key, the referenced row must
exist. 
%
In order to resolve these problems, developers must change the application to
handle these situations and maintain correctness. These solutions also lack
support for users to return.
%
To solve this manually, a developer would have to carefully perform
application-specific database changes to remove data, store any data removed to
be able to later restore it, and correctly revert the database changes on
restoring.
%

%
Second, to manually realized disguised data, developers would also have to
reason about interactions between multiple data-redacting features, and how
these features would compose.
%
For example, imagine an application that supports both account deletion and
anonymizing old data: if a user wants to delete all their posts after they have
been anonymized, a SQL query must somehow determine which anonymized posts
belong to the user in order to remove them.
%
And if the user later wants to return and restore their posts, the developer
must account for the applied anonymization and restore the user's posts as
anonymized.
%

%
Third, stored removed data should be protected against data breaches, but 
accessible if the user chooses to return. The developer also needs to provide
user-friendly ways for users to prove their ownership of the stored data.
%

%
Fourth, application databases often undergo global updates such as schema
migrations and data transformations like content reformatting. However, these
updates would fail to affect any data disguised at the time, because the
disguised data remains inaccessible to the application and developer.  In order
to restore disguised data correctly, these updates must be applied to disguise
data when a user grants permission to access the disguised data in order to
reveal it.
%
For example, a disguised \texttt{posts} row might need to be restored as one 
\texttt{posts} and one \texttt{postContent} row after a schema migration
moves post content into a different table from posts. 
%
The developer thus should remember and apply these global updates to disguised
data in chronological order during reveal.
%


%*********************************************************************************%
%*********************************************************************************%
%*********************************************************************************%
\section{Our Approach}
%
To realize disguised data, we present a general system that helps developers
specify and apply two kinds of transformations: \emph{\xxing transformations},
which move the user's data into a disguised state; and \emph{revealing
transformations}, which restore the original data at a user’s request.
%
\Xxing transformations aim to protect the confidentiality of users' \xxed data
(\eg links to throwaway accounts or old HotCRP reviews) even if the application
is later compromised (\eg via a SQL injection or a compromised admin's account).
%
Only the user can invoke revealing transformations on their disguise data to
authorize the application to restore it.
%

%
We demonstrate our approach in \sys, a system that realizes disguising and
revealing transformations for database-backed web applications via a small set of
expressive primitives that compose cleanly.
\todo{lily got rid of well-defined.}
%
Developers specify the transformations that their application should provide,
and \sys takes care of correctly applying, composing, and optionally reverting
them, while maintaining application functionality and referential integrity.
%

Our approach for \sys's approach faces three challenges.
First, \sys needs to present a simple, yet versatile interface for developers to
specify \xxing transformations.
%
\sys addresses this challenge with a restricted programming model centered
around three primitives: \emph{remove}, \emph{modify}, and \emph{decorrelate} (which reassigns data
to placeholder users).
%
This model limits the potential for developer error, and lets \sys derive the
correct \xxing and revealing operations, while supporting a wide range of
transformations.
%

%
%\head{Pseudoprincipals.}
Second, to work with existing applications in practice, \sys's \xxing
transformations should require minimal application modifications.
%
To achieve this, \sys introduces \emph{pseudoprincipals}, anonymous placeholder
users that are inserted into the database on \xxing and exist solely to own data
decorrelated from real users and maintain referential integrity.
%
Pseudoprincipals can also act as built-in ``throwaway accounts,'' as they let
the user disown data after-the-fact, as well as potentially later reassociate
with it.
%
To correctly reason about ownership when data may be decorrelated multiple times
(\eg by global anonymization after throwaways have been created), \sys maintains
an encrypted \emph{speaks-for chain} of pseudoprincipals that only the original user
can unlock and modify.
%

%
%\head{Diff Records and Reveal Credentials.}
Third, \sys needs to have access to the original data for users to be able to
reveal their data and return to the application, but the whole point is to make
that data inaccessible to the service.
%
While \sys could ask users to store their own \xxed data, this would be
burdensome.
%
Instead, \sys stores the \xxed data on the server in encrypted form as
\emph{diff records}, and unlocks
and restores data to the service only when a user provides their \emph{reveal
credentials} (\eg a password or a private key).

%
\section{Contributions}
%
The main contribution of this thesis is the identification and exploration of
\emph{disguised data} for flexible user data privacy in web applications. As
part of this, this thesis contributes:

\begin{enumerate}[nosep]
    \item The abstraction of \emph{data disguising via \xxing and revealing
        transformations}, including 
        %abstractions for developers and users of web applications, 
        %including \xxing and revealing transformations;
        disguise specifications written with a small, composable set of
        data-anonymizing primitives (remove, modify, and decorrelate to
        pseudoprincipals), and reveal credentials to allow users
        to reveal their data (Ch.~\ref{ch:overview}).
        %, which cover a wide range of application
        %needs and compose cleanly; 

    \item A design for \sys that realizes disguised data for practical web
        applications, based on techniques such as diff records, speaks-for
        chains, and reveal-time update specifications.
        (Ch.~\ref{ch:design}).

    \item A prototype Rust library implementing \sys for MySQL-backed web applications that
        implements user data control via disguising and revealing
        (Ch.~\ref{ch:impl}).

    \item Case studies that integrate \sys with three real-world web
    applications and demonstrate \sys's ability to enable composable and
        reversible transformations (Ch.~\ref{ch:case_studies}).

    \item An evaluation of \sys's effectiveness and performance, including how
    \sys contrasts with and complements related work (Qapla~\cite{qapla} and
        CryptDB~\cite{cryptdb}) (Ch.~\ref{ch:eval}).
\end{enumerate}
%

%\subsection{Limitations}
%
While disguised data can help developers to add more flexible user data
controls, the abstraction and its implementation in \sys have some limitations. 
%
First, disguised data is a concept scoped for single applications, and does not
tackle the problem of data sharing between services.
%
%While \sys enables \xxing and revealing transformations in a broad class of
%applications, \sys has some limitations.
%
\sys also assumes bug-free \xx specifications, and that applications use \sys
correctly.
%
Furthermore, \sys does not aim to protect un\xxed data in the database against compromise;
combining \sys with an encrypted database can add this protection.
%
Finally, attacks to identify users from \sys's metadata (\eg the size of
stored \xxed data) or placeholder data left in the database (\eg embedded text)
are out of scope.
%

The \sys prototype is open-source at \url{https://github.com/tslilyai/edna}.

\section{Related Publications}
The work described in this thesis has been previously covered in two
peer-reviewed publications:

\begin{itemize}
    \item \fullcite{edna-hotos}
    \item \fullcite{edna}
\end{itemize}
%
This thesis contains additional material on handling global database update
operations such as schema migrations and content changes, described in
\S\ref{s:overview:updates}, \S\ref{s:semantics:updates} and \S\ref{s:design:updates}, and evaluated
in \S\ref{s:eval-updates}.

\iffalse
\section{Remaining Work: Revealing with Schema Migrations and Application
Updates}

In \sys's current design, \textbf{global database updates} that implicitly enforce
invariants on application data remain unknown (and thus uncheckable) by \sys.
\sys also fails to reveal disguised data affected by \textbf{schema migrations}
performed since the time of disguise.
%
We will add an API for developers to log important updates and schema migrations with \sys, which \sys will apply to disguised data prior to restoring it to the database.

\sys will apply these updates or schema migrations prior to performing \sys's
existing consistency checks; these checks enable \sys to detect if revealing
transformations will violate referential integrity or other structural database
invariants (\eg uniqueness requirements). 

\fi
