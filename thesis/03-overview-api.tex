\section{\sys API}
\label{s:api}

\begin{figure}[t]
\begin{lstlisting}[style=rust,escapeinside={(*}{*)}]
// Generates keypair for p and returns the user's backup credential
(*\textbf{RegisterPrincipal}*)(
        p: UID, 
        pw: Password,
        pubkey: PublicKey, 
        privkey: PrivKey)
    -> RevealCredential;

// disguises principal p according to the spec, 
// optionally over already-disguised data ((*\S\ref{s:composition}*))
(*\textbf{DisguiseData}*)(
        p: Option<UID>, 
        spec: DisguiseSpec,
        principal_gen: PrincipalGenerator,
        schema: Schema,
        disguise_over: Option<RevealCredential>) 
    -> disguiseID;

// Reveals data disguised by s for p with p's password. 
(*\textbf{RevealData}*)(
    p: Option<UID>, 
    did: disguiseID, 
    pp_ref_policy: PseudoprincipalReferencePolicy,
    allow_partial_row_reveal: bool,
    schema: Schema,
    cred: RevealCredential>);

// Gets principals that p can speak-for.
(*\textbf{CanSpeakFor}*)(p: UID, cred: RevealCredential) -> Vec<UID>;

// Records a reveal-time update spec in the replay log.
(*\textbf{RecordUpdate}*)(update_spec: RevealTimeUpdateSpec);
\end{lstlisting}
\caption{\sys's API (Rust-like syntax).}
\label{f:api-high}
\end{figure}
%

Developers add hooks to their application to invoke \sys's API, which consists
of the following functions (Figure~\ref{f:api-high}). This section describes
each function and how developers would use them to support disguising and
revealing.

\subsection{\texttt{RegisterPrincipal}.}

    Registers an application user to allow their data to be disguised and
    revealed. Unique users should not be registered multiple times.  Once
    registered, data of an undeleted user that exists in the database can always
    be disguised and revealed.

    \paragraph{Arguments.} Takes the user's associated unique identifier and unique
    public-private keypair (generated by the user
    client or application).
    %
    The provided public key is used to encrypt the user's disguised data. The
    provided private key and password enable clients to later reveal data using
    either the private key or password.

    \paragraph{Return Value.} 
    The function returns a backup token that the application should return to
    the user client. This token enables the user to reveal in the case they lose
    their password or private key.


\subsection{\texttt{DisguiseData}.}

    Rewrites and/or removes data from the application database according to the
    provided disguise specification (\ie disguises data). The original data is encrypted
    and stored within the application database; a user must provide credentials
    to invoke with a \texttt{RevealData} call in order to restore the data.
    
    This function may insert pseudoprincipals (anonymous users) into the
    application database if a decorrelation is specified by the disguise.
    \todo{we haven't talked about decorrelation yet?} Data disguised within the
    same disguise may refer to (via a foreign key) the same pseudoprincipal
    specified.
    %
    No two disguises' sets of produced pseudoprincipals overlap. 

    \paragraph{Arguments.} 
    A disguise is invoked either automatically by the application, or a specific
    user of the application identified by user ID. 
    % 
    If disguise is user-specific, then only data that has a foreign key to that
    user's identifier (is ``owned'' by that user), and which matches the
    disguise specification, will be disguised.  Otherwise, all data matching the
    disguise specification will be disguised (potentially owned by multiple
    users). 
    %

    %
    If a user invokes a disguise through the application and provides a reveal
    credential (their password, private key, or backup token), the function will
    disguise data that has already been disguised (\S\ref{s:composition}).
    Depending on the disguise specification, this will remove, modify, or create
    a new pseudoprincipal owner for data with a foreign key to a pseudoprincipal
    instead of the user.
    %

    %
    The developer provides three arguments:
    \begin{itemize}[nosep]
    \item A disguise specification that describes how to disguise data by
    removing, modifying (replacing some or all of its contents with placeholder
    values), or decorrelating, replacing links to users with links to
    pseudoprincipals.
    
    \item A principal generator, which describes how to create a
    pseudoprincipal in the application (global across all disguises).
    
    \item The database schema, which specifies ownership links from data tables to user
    tables via foreign key relationships (global across all disguises).
    \end{itemize}

    \paragraph{Return Value.} 
    The function returns a unique disguise ID for the applied disguise, which
    should be returned to affected users in case they wish to later reveal the
    disguise.

    \paragraph{Special Considerations.}
    If a data row to be removed by the disguise is shared among multiple users,
    the data row is only removed once all users have agreed to disguise. This
    occurs if \one{} all users individually disguise the data (\texttt{p =
    Some(uid)} for all owners \texttt{p}), or \two{} the disguise is applied
    over all users (\texttt{p = None}).
   
    If the data is not removed (some owning users have not disguised the row),
    the data will refer to a pseudoprincipal for users who have disguised.


\subsection{\texttt{RevealData}}

    Restores data disguised by the disguise corresponding to the provided ID to
    the database.
    Revealing the same disguise ID multiple times will do nothing after the
    first reveal.

    If revealing a row violates database constraints specified in the schema
    (\eg uniqueness, referential integrity, NULl checks), \texttt{RevealData} will not
    restore that row.  

    If an attribute of a row has been changed by the application to differ from
    the original value since the row was disguised. This scenario is impossible for
    data that has been removed from the database during the disguise.

    changed since row's original values the
    row will be partially 

    The reveal of non-conflicting rows produced by the same
    disguise may still be restored to the database.

    Revealed data will reflect all of the updates registered via
    \texttt{RecordUpdate} since the time of disguise \texttt{did}, applied in
    chronological order.

    If a removed data row to be restored by the disguise is shared among
    multiple users, ; all other users owning the
    data will remain disguised.


    the data row is only removed once all users have agreed to disguise. This
    occurs if \one{} all users individually disguise the data (\texttt{p =
    Some(uid)} for all owners \texttt{p}), or \two{} the disguise is applied
    over all users (\texttt{p = None}).

    \paragraph{Arguments.} 
    A disguise is invoked either automatically by the application, or a specific
    user of the application identified by user ID. 
    % 
    If disguise is user-specific, then only that user's data that matches the
    disguise specification will be disguised.  Otherwise, all data matching the
    disguise specification will be disguised (potentially data of multiple
    users). 
    %
    If a user invokes a disguise through the application and provides a reveal
    credential (their password, private key, or backup token), the function 
    will disguise data
    that has already been disguised (\S\ref{s:composition}).

    The developer provides three arguments:
    \begin{itemize}[nosep]
    \item A disguise specification that describes how to disguise data by
    removing, modifying (replacing some or all of its contents with placeholder
    values), or decorrelating, replacing links to users with links to
    pseudoprincipals.
    
    \item A principal generator, which describes how to create a
    pseudoprincipal in the application (global across all disguises).
    
    \item The database schema, which specifies ownership links from data tables to user
    tables via foreign key relationships (global across all disguises).
    \end{itemize}

    \paragraph{Return Value.} 
    The function returns a unique disguise ID for the applied disguise.




\iffalse
%
(1) An application registers users with a public--private keypair
that either the application or the user's client generates; \sys stores the
public key in its database, while the user retains the private key for use in
future reveal operations.
%

%
(2) When the application wants to \xx some data, it invokes \sys with the
corresponding developer-provided \xx specification and any necessary
parameters (such as a user ID).
%
\Xx specifications can remove data, modify data (replacing some or all of its
contents with placeholder values), or decorrelate data, replacing
links to users with links to pseudoprincipals (fake users).
%
% Decorrelation preserves the structure of the application database, and avoids
% integrity issues like dangling foreign keys, while obscuring the data's
% relationship to natural principals (true users).
%
\sys takes the data it removed or replaced and the connections between
the user and any pseudoprincipals it created, encrypts that data with the user's
public key, and stores the resulting ciphertext---the \emph{\xxed
data}---such that it cannot be linked back to the user without the user's
private key.
%
%The application's database now no longer contains the \xxed data.
%


%
(4) When a user wishes to reveal their \xxed data, they pass credentials
to the application, which calls into \sys to reveal the data.
%
Credentials are application-specific: users may either provide their private
key or other credentials sufficient for \sys to re-derive the private key.
%
\sys reads the \xxed data and decrypts it, undoing the changes to the
application database that \xxing introduced.
%

%
\sys provides the developer with sensible default \xxing and revealing
semantics (\eg revealing makes sure not to overwrite changes made since
\xxing).
\fi
