\todo{Update with numbers from Google Cloud}
\todo{Fix figures}

Our evaluation seeks to answer six questions:
%
\begin{enumerate}[nosep]
 %
 \item How much developer effort and application modification does \sys require? (\S\ref{s:eval-effort})
%
\item How expensive are common application operations, as well as
  \xxing, revealing, and operations over \xxed data with \sys? (\S\ref{s:eval-ops})
 %

\item What overheads does \sys impose, and where do they come from?
    (\S\ref{s:eval-overheads})

\item 
    How does the effort required to implement \sys's
    functionality in a related system (Qapla~\cite{qapla}),
    and its performance, compare with using \sys?
    (\S\ref{s:eval-qapla})

\item 
    What is the performance impact of composing \sys's guarantees
    with those of encrypted databases?
        (\S\ref{s:eval-cryptdb})

\item 
    Which categories of application updates and schema migrations can \sys
        support, and with what overheads? (\S\ref{s:eval-updates})
%
\end{enumerate}

We compare Edna to three alternative settings:
\one{} a manual version
of each \xxing transformation that directly modifies the database
(\eg via SQL queries that remove data), which
lacks support for revealing and does not support
composition of multiple transformations;
\two{} an implementation of disguising and revealing in Qapla~\cite{qapla}
using Qapla’s query rewriting and access control policies; and \three{} an integration of Edna with CryptDB~\cite{cryptdb}, an encrypted database.

%
%To understand the cost of using \sys, we implemented a manual version of each
%\xxing transformation that directly modifies the database, akin to what
%developers might do today (\eg via SQL queries that remove data).
%%
%This manual baseline lacks support for revealing, and does not support
%composition of multiple transformations.
%%

%
%To evaluate the benefits of \sys's design, we compare it to
%Qapla~\cite{qapla}.
%%
%Qapla enforces access control policies by rewriting SQL queries,
%with predicates that restrict the rows returned,
%and we implemented \xxing and revealing functionality using these policies.
%
%With Qapla,
%developers can write policies that enable users to control visibility of their
%data.
%, and developers may want to use Qapla to implement user data controls.
%
%Finally, we combine \sys with CryptDB~\cite{cryptdb} and show that the two
%systems offer different, complementary guarantees.
%

All benchmarks run on a Google Cloud \texttt{n1-standard-16} instance with 16 CPUs
and 60 GB RAM, running Ubuntu 20.04.5 LTS. Benchmarks run in
a closed-loop setting, so throughput and latency are inverses. %of each other.
%
We use MariaDB 10.5 with the InnoDB storage engine atop a local SSD.
%

%%%%%%%%%%%%%%%%%%%%%%%%%%%%%%%%%%%%%%%%%%%%%%%%%%
\section{\sys Developer Effort}
\label{s:eval-effort}

\begin{figure}[t]
\centering
\small
\begin{tabular}{c|ccc}
    \textbf{App} & \textbf{Original LoC} & \textbf{Spec LoC (JSON)} & \textbf{Hooks LoC}\\
\hline
    Lobsters & 160k & 518 & 179 \\
    WebSubmit & 908 & 75 & 312 \\
    HotCRP & N/A & 357 & N/A \\
\end{tabular}
    \caption{Adding \sys specs (JSON) and \xxing/revealing endpoints require
    reasonable developer effort.}
  \label{f:loc}
\end{figure}

% We estimate developer effort to use \sys via the lines of code of spec (JSON)
% and application code added to support \xxing and revealing transformations.

%Although developer effort is difficult to quantitively measure, there are
%clear effort benefits to using \sys over a manual approach, which we now engage with.
%
We evaluate the developer effort required to use \sys by measuring the
difficulty of implementing the \xxing and revealing transformations in our three
case studies.  This took one person-day per case study for a developer
familiar with \sys but unfamiliar with the applications.

A developer supporting these transformations must first add application
infrastructure to allow users to invoke them and notify users when they happen.
This is required even if the developer were to implement transformations
manually without \sys.
%Any support for these transformations---even if implemented manually without
%\sys---requires application changes to allow users to invoke them, and notify
%users when they happen.
%
%Support for \xxing and revealing transformations---even if implemented manually
%without \sys---necessarily require changes to application code. These changes
%add HTTP endpoints to invoke \xxing and revealing, and modifications to perform
%authorization checks for anonymous users and send emails to users with \xx IDs.
These changes add 179 LoC of Ruby to Lobsters (160k LoC), and 312 LoC of Rust
to the original WebSubmit (908 LoC). They implement HTTP endpoints,
authorization of anonymous users, and email notifications.
%

%
A developer using \sys also writes \xx specifications and invokes \sys.
Lobsters' \xx specifications are written in 518 LoC, WebSubmit's in 75 LoC, and
HotCRP's in 357 LoC (all in JSON).  The specification size is proportional to
schema size and what data each application \xxs.
%

%
Our Rust/SQL implementation of the three Lobster bulk database management
operations required \todo{xxx} LoC, and the corresponding updates for
disguised data required \todo{xxx} additional LoC. \todo{check}
%

Thus, the developer effort required to use \sys---writing \sys specifications, and
invoking \sys---is small, even though these applications were not written with
\sys in mind.

%%%%%%%%%%%%%%%%%%%%%%%%%%%%%%%%%%%%%%%%%%%%%%%%%%%%%%%%%%%%%%%%%%%%%%%%%%%%%%%%%%%
\section{Performance of \sys Operations}
\label{s:eval-ops}

%
We now evaluate \sys's performance using WebSubmit,
HotCRP, and Lobsters (\S\ref{s:case-studies}).
%
%To measure the extra cost that \sys's cryptographic operations add,
%, as a developer would do without \sys.
%, and measure its latency.
%
We measure the latency of common operations, \xxing transformations,
and operations over \xxed data enabled by \sys
(\eg account restoration and editing \xxed data).
%
The three applications do not create new data that references pseudoprincipals,
but to fully capture any overheads we configure \sys to nevertheless run the
checks for lingering pseudoprincipal references on revealing.
%
A good result for \sys would show no overhead on common operations,
competitive performance with manual \xxing, and reasonable
latencies for revealing operations only supported by \sys
(\eg a few seconds for account restoration)
%impossible in the manual baseline
%

\begin{figure}
\begin{subfigure}[b]{\columnwidth}
    \centering
  \includegraphics[width=.8\columnwidth]{figs/websubmit_op_stats}
\caption{WebSubmit (2k users, 80 answers/user).}
\label{f:ops-websubmit}
\end{subfigure}
\begin{subfigure}[b]{\columnwidth}
    \centering
    \includegraphics[width=.8\columnwidth]{figs/hotcrp_op_stats}
  \caption{HotCRP (80 reviewers, 3k total users, 200--300 records/reviewer).}
\label{f:ops-hotcrp}
\end{subfigure}
\begin{subfigure}[b]{\columnwidth}
    \centering
    \includegraphics[width=.8\columnwidth]{figs/lobsters_op_stats}
\caption{Lobsters (16k users, Zipf-distributed data/user).}
\label{f:ops-lobsters}
\end{subfigure}
\caption{\sys adds no latency overhead to common application operations and
modestly increases the latencies of \xxing operations compared to a manual
implementation that lacks support for revealing or composition.
%
Bars show medians, error bars are 5\textsuperscript{th}/95\textsuperscript{th}
percentile latencies.}
\label{fig:client_opstats}
\end{figure}

\textbf{WebSubmit.}
%
We run WebSubmit with a database of 2k users, 20 lectures with four questions
each, and an answer for each question for each user (160k total answers).
%We implement WebSubmit account removal and answer anonymization both manually
%as an irreversible database change, and with \sys.
%
We measure end-to-end latency to perform common application operations (which
each issue multiple SQL queries), as well as \xxing and revealing operations
when possible (revealing operations are impossible in the baseline).
%
%A good result for \sys would show competitive performance with the manual
%baseline, as \sys does strictly more work (\eg encrypting \xxed
%data).
%
%Latency includes request processing in WebSubmit and \sys's
%operations, and the response to the client.
%
Figure~\ref{f:ops-websubmit} shows that common operations have comparable
latencies with and without \sys.
%
\sys adds 9ms to account creation; and \xxing
and revealing operations take longer in \sys (13.1--53.2ms), but allow users to
reveal their data and take less developer effort.
%

\textbf{HotCRP.}
%
We measure server-side HotCRP operation latencies for PC members on a database seeded with 3,080
total users (80 PC members)
%and per-user data such as paper watches;
and 550 papers with eight reviews, three comments, and four conflicts each
(distributed evenly among the PC).
%
HotCRP supports the same \xxing transformations as WebSubmit, but PC users have more
data (200--300 records each), and HotCRP's \xxing transformations mix deletions and
decorrelations across 12 tables. %, so we would expect higher latencies.
%

%
Figure~\ref{f:ops-hotcrp} shows higher latencies in general, even for the manual
baseline, which reflects the more complex \xxing transformations.
%
\sys takes 63.8--84.6ms to \xx and reveal a PC member's data, again owing to the
extra cryptographic operations necessary.
%
HotCRP's account anonymization is admin-applied and runs for all PC members, so
its total latency is proportional to the PC size.
%
With 80 PC members, this transformation takes 6.8s, which is acceptable for a
one-off operation.
%
As before, \sys adds small latency to common application operations, and 9ms to
account creation.
%

\textbf{Lobsters.}
%
We run Lobsters benchmarks on a database seeded with 16k users, and
120k stories and 300k comments with votes, comparable to the late-2022 size of
production Lobsters~\cite{lobsters}.
%
Content is distributed among users in a Zipf-like distribution according to
statistics from the actual Lobsters deployment~\cite{lobsters-data}, and 20\% of
each user's contributions are associated with the topic to anonymize.
%
The benchmark measures server-side latency of common operations and
\xxing/revealing transformations.
%
%Because Lobsters users have far more variable amounts of
%data, we expect higher variability in latencies.
%and we measure the account
%decay \xxing transformation (and subsequent restoration) in addition to GDPR-compliant
%account removal and restoration.
%
%Lobsters does not support editing decayed contributions.
%
%As the amount of data per user follows a Zipf-like distribution, we would
%expect \xxing some users to be more expensive than others.
%

%
The results are in Figure~\ref{f:ops-lobsters}.
%
The median latencies for entire-account removal or decay are small (9.7--13.4ms
for \sys, and 4.0--5.2ms for the baseline), since the median Lobsters user has
little data. Revealing \xxed accounts takes 13.1--17.6ms in the median.
%
Highly active users with lots of data raise the 95\textsuperscript{th} percentile
latency to 100--180ms for \xxing and 45--80ms for revealing.
%
Topic anonymization touches less data and is faster than
whole-account transformations, taking 3.6ms and 13.1ms for the median user to
\xx and reveal, respectively.
%

\textbf{Summary.}
%
\sys necessarily adds some latency compared to manual, irreversible data
removal, since it encrypts and stores \xxed data.
%
However, most \xxing transformations are fast enough to run interactively as
part of a web request.
%
Some global \xxing transformations---\eg HotCRP's conference anonymization over
many users---take several seconds, but an application can apply these
incrementally in the background, as in Lobsters account decay.
%
%We next break down these costs.
%%%%%%%%%%%%%%%%%%%%%%%%%%%%%%%%%%%%%%%%%%%%%%%%%%%%%%%%%%%%%%%%%%%%%%%%%

\subsection{\sys Performance Drill-Down}
\label{s:eval-additional}
%\lyt{made this a subsubsection}

\begin{figure}[h]
    \centering
    \includegraphics[width=\columnwidth]{figs/composition_legend}
    \begin{subfigure}[b]{.48\columnwidth}
        \includegraphics[width=\columnwidth]{figs/composition_stats_websubmit}
        \caption{WebSubmit}
        \label{f:comp-websubmit}
      \end{subfigure}
      \begin{subfigure}[b]{.48\columnwidth}
          \includegraphics[width=\columnwidth]{figs/composition_stats_hotcrp}
          \caption{HotCRP}
        \label{f:comp-hotcrp}
      \end{subfigure}
    \caption{Applying \xxing transformations to previously-decorrelated accounts
      increases latency linear in the number of pseudoprincipals involved.
      %to the amount of data touched, as \sys performs
      %3--5.5$\times$ and restoration by 5.7--10$\times$ as \sys
      %cryptographic operations linear in the number of pseudoprincipals
      %involved.
      Hatched lines indicate the proportion of cost attributed to
      cryptographic operations.}
    \label{f:composition}
\end{figure}

We next break down the cost of \sys's operations into the cost of database
operations and the cost of cryptographic operations.
%
%\sys's basic API performs some computation (which we expect to be quick) as
%well as two operations that could cause bottlenecks, namely data encryption
%and decryption and access to \sys's database.
%
%\sys's high-level API additionally performs queries on the application database.
%
\lyt{Each disguise and reveal operation requires decoding of JSON specifications
into \sys-specific Rust datatypes; this takes $\approx0.1$ms on average. This
could be offloaded to registering each specification with \sys when the
application starts, allowing \sys to decode them in advance.}
%
\sys's database operations are fast; in our prototype, they generally take
$0.2$--$0.3$ms but vary depending on the amount of data touched.
%
%Database query latency depends on the amount of data
%touched, but is normally $<$1ms in our case studies.
%
\sys's cryptographic operations are comparatively expensive.
%; Figure~\ref{f:opstats} shows their cost.
%
PBKDF2 hashing for private key management incurs a 8ms cost and affects account
registration and operations on \xxed data that reconstruct a user's private key;
this accounts for up to 79\% of these operations' cost when the operation issues
only a few database queries.
%
%This appears during registration (generating password-based credentials from the
%private key), as well as operations on \xxed data that reconstruct the private key
%(\ie editing or restoring \xxed data).
%
%Applications that only wish to support private keys as reveal credentials can tell
%\sys to skip PBKDF hashing and secret sharing, eliminating this cost.
%

Encryption and decryption incur baseline costs of 0.1ms and 0.02ms respectively;
their cost grows linearly with data size.
%
In the common case, \xxing or revealing data performs two cryptographic
operations: one to encrypt/decrypt the diff and speaks-for records, and one to
encrypt/decrypt the ID at which they are stored.
%

\sys also generates a new key for each pseudoprincipal created, which takes
0.2ms.
\sys's cryptography accounts for up to 35\%
%1--35\%
of the cost of \xxing/revealing operations such as account removal or
anonymization; this proportion decreases as the number of database
modifications made by a transformation increases.
%Disguising undisguised data performs a single encryption, and
%A new disguise or reveal operation requires a single encryption/decryption
%when a single disguise to the data.
%
When the application applies multiple \xxing transformations and \xxs the
data of pseudoprincipals, doing so may require several encryptions/decryptions.
%
We evaluate this cost next.
%

%
%\lyt{ADDED THIS.}
%%
%\sys's \xxing transformations (\eg remove account) incur costs proportional to
%the amount of data a user owns: decorrelating a user's per-lecture answers in
%WebSubmit, for example, requires retrieving the user's answers,  inserting
%per-lecture pseudoprincipals, and update queries (one per each of the 80
%answers) to rewrite answers to correlate to pseudoprincipals. This results in a
%total of $\approx$18ms simply to perform database modifications.
%%
%For each of these decorrelations, \sys produces speaks-for records that are then
%encrypted and persisted, accounting for the remaining 2ms costs.
%%
%
%The cost of \sys's revealing transformations (\eg restore account) can also be
%explained as a breakdown between database query overheads and decryption
%overheads.
%%
%Restoring removed answers first requires decrypting a user's diff tokens.
%%
%For each of a user's 80 answers, \sys performs three selection queries for
%consistency checks to ensure that reinsertion can correct proceed (0.4-0.5ms)
%and inserts the answer (0.2ms).
%%
%Total restoration of a user's answers thus takes $\approx$50ms; the remaining 1-2ms cost
%comes from clearing the diff and speaks-for tokens of the revealed
%transformation, which requires both a decryption and a database deletion query.
%%
%Wrapping all reveal queries in a transaction did not meaningfully improve
%performance.
%%
%

\subsection{Composing \Xxing Transformations.}
\label{s:eval-composition}
% \lyt{this is now under \sys op performance}.

%
To understand the overhead of composing transformations in \sys,
we measure the cost of composing account removal on top of
%from an application that already invoked
a prior \xxing transformation to anonymize and decorrelate all users' data.
%
We consider WebSubmit and HotCRP, and compare three setups: \one{} manual
account removal (as before); \two{} account removal and restoration
\emph{without} a prior anonymization \xxing transformation; and \three{} account
removal and restoration \emph{with} a prior anonymization \xxing transformation.
%
With prior anonymization, a subset of the user's data has already been decorrelated
when removal occurs, and removal therefore performs per-pseudoprincipal encryptions of
\xxed data with pseudoprincipals' public keys.
%by the first \xxing transformation, and \sys
%
%This requires individually encrypting data with each pseudoprincipal's
%public key.
%
Restoring the removed, anonymized account must then individually decrypt
pseudoprincipal records and restore them.
% to both restore and then clear them.
%
Hence, \xxing and revealing in the third setup should take time proportional to
the number of pseudoprincipals created by anonymization.
%

%
Figure~\ref{f:composition} shows the resulting latencies.
%
WebSubmit account removal and restoration latencies increase by $\approx$1ms per
pseudoprincipal (18.2ms and 21.8ms respectively); 50\% of this increased cost
comes from the additional, per-pseudoprincipal encryption and decryption of
records, the rest comes from database operations.
%
HotCRP removal and restoration latencies also increase by $\approx$1ms
per pseudoprincipal (191.2ms and 230.4ms respectively); again, cryptographic
operations add $\approx$0.5ms per pseudoprincipal, and the remaining cost
increase comes from per-pseudoprincipal database queries and
updates.
%querying-for and per-pseudorpicnipal
%
WebSubmit and HotCRP do not create new references to pseudoprincipals
after data gets disguised, but if they did, \sys would need to issue additional
per-pseudoprincipal queries to rewrite or remove these references (if configured
to do so).
%
Compared to accounts in WebSubmit, accounts in HotCRP have more data and
14--15$\times$ more pseudoprincipals after anonymization, which accounts for the
larger relative slowdown.
%
%HotCRP particularly demonstrates how cryptographic operations grow comparatively
%more expensive: from 21\% of the cost to 41\% for account removal, and from 8\%
%to 25\% for account restoration.
%
%
%Restoring an account (\ie undoing a composed disguise) is more expensive than removing an
%account (\ie composing a disguise atop another) because
%decryption is more expensive than encryption (Figure~\ref{f:opstats}).
%

%
Importantly, \xxing latencies stabilize when \sys composes further \xxing
transformations: since cost is proportional to the number of pseudoprincipals
affected, latency does not grow once the application has maximally decorrelated
data (to one pseudoprincipal per record), as done by HotCRP anonymization.
%\lyt{XXX} HotCRP's anonymization
%performs maximal decorrelation, and the subsequent account removal demonstrates
%this maximal cost.
%


% %
% The experiment composed an interactive \xxing transformation (account removal) atop a
% non-interactive one (anonymization).
% %
% This benefits from the optimization for interactive \xxing transformations mentioned in
% \S\ref{s:composition}: the natural principal introduces locators for
% pseudoprincipal-owned bags that \sys forgot, which helps avoid a locator
% decryption on revealing the composed \xxing transformation.
% %
% Composing a non-interactive \xxing transformation requires this extra encryption/decryption
% as \sys store the locators; but the cost remains linear in the number of
% pseudoprincipals touched.
% %

\section{\sys Overheads}
\label{s:eval-overheads}

\sys adds both space and compute overheads to the application; we measure the
impact of these next.

\subsection{Space Used By \sys.}

%\begin{figure*}[t]
%\centering
%\begin{tabular}{ccc}
%    & \textbf{Pre-Delete (MB)} & \textbf{Post-Delete (MB)} \\
%\hline
%    App DB (does not include metadata) & 261 & 290\\
%    Encrypted Bags (Mem) & 0 & 34.3\\
%    Encrypted Locators (Mem) & 0 & 1.9\\
%    Principal Metadata (Mem) & 1.8 & 16.5\\
%    Shares Metadata (Mem) & 2.0 & 2.0\\
%    Encrypted Bags (Disk) & 0 & 35.1\\
%    Encrypted Locators (Disk) & 0 & 0.4\\
%    Principal Metadata (Disk) & 2.6 & 14.1 \\
%    Shares Metadata (Disk) & 8.9 & 8.9\\
%\end{tabular}
%    \caption{Amount of data before and after 10\% of users remove their accounts. Edna adds an extra
%    110MB on disk (and 44MB in memory).}
%  \label{f:storage}
%\end{figure*}
%
To understand \sys's space footprint, we measure the size of all data stored
on disk by \sys before and after 10\% of users in Lobsters (1.6k users)
remove their accounts.
%
%
Cryptographic material adds overhead and each generated pseudoprincipal adds an
additional user to the application database; \sys also stores data for each
registered principal (a public key and a list of opaque indexes) as well as
encrypted records.
%
%Clients keep track of their credentials, but these are small and encoded
%in application-specific ways.


%
\sys's record storage uses 12 MB, which grows to
58.5 MB after the users remove their accounts, and the application database
size increases from 261 MB to 290 MB (+11\%).
%
(\sys also caches some of this data in memory.)
%
The space used is primarily proportional to the number of pseudoprincipals
produced: each pseudoprincipal requires storing an application database record, a
speaks-for record, and row in the principal table.
%
In this experiment, Lobsters produces 78.1k pseudoprincipals.
%
%Storing a private key is necessary if the application wants to further
%disguise decorrelated data, but avoidable if disguises cannot compose.
%
%In addition, each pseudoprincipal inserts a public key into \sys's
%metadata, adding $\approx$0.3 KB per pseudoprincipal and 24 MB combined,
%
%
\sys removes the public keys and database data for the 1.6k removed principals, but
stores encrypted diff records with their information, which uses another 2.2 MB.
%
%These diff records amount to a small overhead (2.2MB).
%
%2.2MB is 0.8\% of the original Lobsters database size of 261 MB.

%%%%%%%%%%%%%%%%%%%%%%%%%%%%%%%%%%%%%%%%%%%%%%%%%%%%%%%%%%%%%%%%%%%%%%%%
\subsection{Impact On Concurrent Application Use.}
\label{s:eval-conc}

%
%Web applications serve many concurrent users, most of whom are simply using the
%application.
%
%Occasionally, a user (or an admin, or a cron job) will initiate a \xxing
%transformation via \sys.
%
For \sys to be practical, the throughput and latency of normal application requests
by other users must be largely unaffected by \sys's \xxing and revealing operations.
%
%If an expensive \xxing transformation---such as a popular Lobsters user removing their
%account---blocked application processing for all other users, developers would
%be rightly leery of \sys.
%

\begin{figure}[t]
    \centering
    \includegraphics[width=\columnwidth]{figs/lobsters_concurrent_results}
    \caption{Continuous \xxing/revealing operations in Lobsters
    have a <7\% impact on application request
    throughput when disguising a random user; an extreme case of a
    heavy-hitter user with lots of data repeatedly \xxing and revealing
    causes a 3--17\% drop in throughput.}
    \label{f:concurrent-lobsters}
\end{figure}

%
We thus measure the impact of \sys's operations on other concurrent requests
in Lobsters.
%
In the experiment, a set of users make continuous requests to the
application that simulate normal use, while another distinct set of users
continuously remove and restore their accounts.
%
\sys applies \xxing transformations sequentially, so only one transformation
happens at a time.
%
% (This is realistic, as the UI can process later requests asynchronously; it also
% avoids \sys having to reason about overlapping data between two or more \xxing
% transformations.)
%
We measure the throughput of ``normal'' users' application operations, both
without \sys operations (the baseline) and with the application continuously
invoking \sys.
%
The Lobsters workload is based on request distributions in the real
Lobsters deployment~\cite{lobsters-data}.
% users read stories, load the frontpage,
% vote and comment, and post stories.
%

%
Since users' \xxing/revealing costs vary in Lobsters, we measure the
impact of \one{} randomly chosen users invoking account removal/restoration, and
\two{} the user with the most data continuously removing and restoring their
account (a worst-case scenario).
% , since this overlaps most with the data touched by normal users).
%
We show throughput in a low load scenario ($\approx$20\% CPU load),
and a high load scenario ($\approx$95\% CPU load).
%
Finally, we measure settings with and without a transaction for \sys
transformations.
%
%The latter makes sense if the application wishes to avoid exposing
%partially-\xxed data to other clients.
%
A good result for \sys would show little impact on normal operation throughput
when concurrent \xxing transformations occur.
%

%
Figure~\ref{f:concurrent-lobsters} shows the results.
%
If a random user disguises and reveals their data (the common case),
normal operations are mostly unaffected by concurrent \xxing and revealing:
throughput drops $\le$3.7\% without transactions and $\le$7.0\% with transactions.
%
Constantly \xxing and revealing the user with the most data (the worst-case
scenario) has a larger effect, with throughput reduced by up to $7.4$\%
(without transactions) and up to $17$\% (with transactions, high load).
%5.8\% in low load, 2.6 high
%13.3\% in low load, 16.3 high
%dropping by at most 4.6\%---by concurrent \xxing of a random user
%(with and without transactions) and of the most expensive user (without
%transactions); \xxing the most expensive user with transactions does, however,
%impact performance by 11.9\%.
%%
%Under high load, the relative impacts are similar: the throughput when
%concurrently \xxing a random user (with and without transactions) and the most
%expensive user (without transactions) drops by $\le 4.4$\%; and when disguising the
%%most expensive user with transactions, drops by 17.1\%.
%
%\lyt{XXX Note: drop w/out txn is actually more for low load than high load.}
%10% low load
%
%The throughput of normal application requests under low load, with account
%removals and restorations of either random users or the most expensive user,
%remains relatively unaffected at 1.3k ops/s (decreasing by $<$100 ops/s, or
%6\%).
%%
%Account removal and restoration under high load results in a slight decrease
%(1.8\%) in throughput (transactions decrease throughput by 3.8\%), but
%throughput remains high ($>5$ops/ms).
%

This shows that \sys's \xxing and revealing transformations have acceptable
impact on other users' application experience in the common case.
%and operations over \xxed data
%
%\ms{Did we lose a sentence about account removals under low load here? The following doesn't quite flow from the prior point}
%
%Account removals and restorations under low load take longer, with
%the expensive user's account removal and revealing taking 4.4 and 3.6 seconds
%respectively.
%

The latency of disguising operations depends on load:
the expensive user's account removal and revealing take 4.4 and 3.6 seconds
under high load, and 3.3 and 2.6 seconds under low load.
%
This is acceptable: 50\% of data deletions at Facebook take five minutes or
longer to complete~\cite{delf}.
%

%low load
%-0.03667745415
%-0.07011866235
%-0.07443365696
%-0.1488673139

%high load
%-0.03243377074
%-0.04431790047
%-0.02995791037
%-0.1705867789
%cheap disguise 17 27 39
%cheap restore 118 291 472
%cheap disguise txn 22 35 49
%cheap restore txn 118 288 467
%expensive disguise 3212 3517 3693
%expensive restore 2693 2811 2979
%exp txn delete 3029 3269 3481
%exp txn restore 2422 2612 2868
%expensive disguise 4105 4373 4594
%expensive restore 3704 3903 4317
%exp txn delete 4135 4400 4592
%exp txn restore 3542 3661 3785

%%%%%%%%%%%%%%%%%%%%%%%%%%%%%%%%%%%%%%%%%%%%%%%%%%%%%%%%%%%%%%%%%%%%%%%%
\section{Comparison to Qapla}
\label{s:eval-qapla}

This section compares \sys's performance and the effort to use \sys to an
implementation of the same disguising and revealing functionality for WebSubmit
using Qapla’s query rewriting and access control policies. 
%
Unlike \sys, Qapla's goal is not disguised data, but rather to enforce
developer-specified data access policies.
%
This section explores the benefits of \sys---a design specifically tailored for
disguising and revealing---compared to a design that modifies an existing system
(\ie Qapla) to support disguised data.
%

%
\subsection{Effort.}
%
Specifying \xxing transformations as Qapla policies requires more
explicit reasoning about transformations' implementations and their
compositions.
%
In Qapla, a developer would realize \xxing transformations via metadata flags
that they add to the schema (\eg \fn{is\_deleted} for removed data) and toggles in
application code. They then provision Qapla with a predicate that checks if
this metadata flag is \fn{true} before returning a row.
%
Developers must carefully craft Qapla's predicates, which grow in complexity
with the number of \xxing transformations that can compose. For example, an
application supporting both account removal and account anonymization must
combine predicates such that removal always takes precedence. Each additional
transformation increases the number of predicates whose combinations the
developer must reason about.
%
Developers must also optimize Qapla predicates (\eg reducing joins, adding
schema indexes and index hints) to achieve reasonable performance
(\S\ref{s:eval-qapla-perf}).

%
To modify data, the application developer can use Qapla's ``cell blinding''
mode, which dynamically changes column values (to fixed values) based on a
predicate before returning query results.
%
The developer must manually implement more complex modifications and
decorrelation (\ie creating pseudoprincipals and rewriting foreign keys).
%

%
Realizing WebSubmit transformations in Qapla required 576 lines of C/C++, and
110 lines of Rust to add pseudoprincipal, modification, and decorrelation
support.
%(as well as the required 300 lines of Rust code to add
%HTTP endpoints).
%

Overall, Qapla requires more developer effort than \sys, particularly in writing
composable and performant predicates, and manually implementing modifications
and decorrelations. However, Qapla's approach does make some things easier.
%
Because data remains in the database, revealing simply requires toggling
metadata flags, and data to reveal can adapt to database changes (\eg
schema updates). But keeping the data in the database also means that developers cannot use Qapla to
achieve GDPR-compliant data removal. %, as the data remains in the database.

%
\begin{figure}[t]
  \centering
      \includegraphics[width=\columnwidth]{figs/websubmit_qapla_op_stats}
    \caption{\sys achieves competitive performance with a manual baseline and
    outperforms Qapla on nearly all common WebSubmit operations (2k users,
    80 answers/user).
    Bars show medians, error bars are 5\textsuperscript{th}/95\textsuperscript{th}
    percentile latencies.}
  \label{f:qapla_ws_opstats}
\end{figure}

%
\subsection{Performance.}
\label{s:eval-qapla-perf}
% comparable with \sys.
%
We measure Qapla's performance (Figure~\ref{f:qapla_ws_opstats}) on the same
WebSubmit operations (Figure~\ref{f:ops-websubmit}).
%
Qapla performs well on operations that require only writes, since Qapla does not
rewrite write queries.  Removing and restoring accounts requires only a single
metadata flag update in Qapla, whereas \sys encrypts/decrypts user data and
actually deletes it from the database.
%
However, Qapla rewrites all read queries, so Qapla performs poorly on operations
that require reads, such as listing answers and editing (disguised or
undisguised) data.
%
Qapla's query rewriting takes $\approx$1ms, and rewrites \fn{SELECT} queries in
ways that affect performance (\eg adding joins to evaluate predicates).
%
Overall, \sys achieves better performance on common operations.

%
%Furthermore, \sys achieves better performance than Qapla on common operations.


\input{07-eval-cryptdb}
%%%%%%%%%%%%%%%%%%%%%%%%%%%%%%%%%%%%%%%%%%%%%%%%%%%%%%%%%%%%%%%%%%%%%%%%

\section{Database Management Operations}
\label{s:eval-updates}

%
If developers perform database management operations, this both increases
developer effort and affects \sys's performance. Here, we answer the following two questions:
\begin{enumerate}[nosep]
    \item What categories of bulk management operation can \sys support, and how
        easily? 
    \item What are the performance impacts of invoking \sys's API to record an
        update, and applying updates during reveal?
\end{enumerate}

\subsection{Supporting Updates}
\begin{figure}
    \centering
    \small
    \begin{tabular}{m{0.28\columnwidth}|m{0.2\columnwidth}|m{0.2\columnwidth}|m{0.2\columnwidth}}
        \centering\textbf{Attribute} & \textbf{Normalize URL} & \textbf{Add Story Texts} &
        \textbf{Add Show Email} \\
        \hline
        Impl Bulk Op & 68 LoC & 19 LoC & 2 LoC \\
        \hline
        Impl Update & $\varnothing$ & 54 LoC & 27 LoC \\
        \hline
        Perform Bulk Op & 5.63s & 5.10s & 0.095s\\
        \hline
        Perform Update (story row) & 0.22ms & 0.01ms & 0.01ms \\
        \hline
        Perform Update (user row) & <0.01ms & 0.01ms & 0.01ms \\
        \hline
        Perform Update (other row) & <0.01ms & <0.01ms & <0.01ms \\
    \end{tabular}
    \caption{Schema migrations that affect entire tables require
    writing separate update function for \sys. Applying these updates adds greater overheads per
    story than per user or per other table row, as more updates apply to the
    stories table.}
    \label{tab:updates}
\end{figure}

Developers must already write their management operations today (\eg Lobsters
implements these updates as Ruby Active Record Migrations~\cite{ruby_arm}). 
%
However, depending on the type of operation, developers may need to additionally
write an update specification for \sys to capture that operation.
%
Operations that execute by affecting entire tables (\eg via an \texttt{ALTER
TABLE} command) require writing separate updates which take as input a set of
rows instead of an entire table.
%
However, bulk management operations may already operate on a row-level, and thus
can be reused as an update specification to \sys.


%
Our Lobsters implementation implements the selected three bulk operations in
Rust instead of Ruby Active Record Migration. Table~\ref{tab:updates} shows the
amount of code required per operation, as well as the additional code required
to add a corresponding update to give \sys. We reuse the bulk URL normalization
operation for the update to \sys. We implemented the other two schema migrations
as bulk table updates, however, and thus wrote separate updates that operate on
individual rows.
%
The additional lines of code to write the updates on top of the bulk operation
are either 0 when the update acts as a subroutine of the bulk operation, or
less than 100 LoC.
\todo{too vague? it seems a lot compared to the table sql statements, but
updates are rather straightforward.}
%
Invoking \sys with the three different application updates and schema migrations
for Lobsters required adding one line of code per update that the application
calls when performing the bulk operation. 
%

\begin{figure}
\centering
  \includegraphics[width=.7\columnwidth]{figs/categories}

    \caption{Out of the most recent 20 Lobsters database management operations,
    the majority (75\%) are table operations which require developers to write
    separate update specifications for \sys, 15\% are row operations which can
    be reused as updates without additional effort, and 10\% are unsupported
    because they denormalize the schema in order to cache aggregate results.}
  \label{tab:categories}
\end{figure}

%
In addition to the updates described and implemented in
\S\ref{s:casestudies:updates}, we inspected the 20 most recent management operations in
Lobsters from the past three years, and classified them as:
\begin{enumerate}[nosep]
%
    \item a row operation (which can be reused as an update to \sys), such as URL
normalization; 
%
\item a table operation (which requires writing a new update for \sys), such as
generating tables from other tables (\eg texttt{story\_texts} from
\texttt{stories}) or altering columns; or
%
\item an unsupported operation (which cannot be supported as an update with
    \sys), such as updating the \texttt{vote\_count} of comments based on the number of votes).
\end{enumerate}
Figure~\ref{tab:categories} shows the distribution of 20 bulk operations among these
three categories. 
%
\sys can support all operations other than updating rows based
on current external database state, unless the update's output is correct no
matter the state of other database tables. This represents 2 of the 20 inspected
updates.
%
%
%As described in \S\ref{s:design:limits}, unless it is valid application behavior
%for \eg \texttt{vote\_count} to be 0 regardless of the number of votes, \sys
%will not correctly apply an update that sets the count of votes for comments.
%
These types of operations might occur in applications that optimize for
performance by denormalizing their database schema; we hypothesize that this
category of update would not exist with a normalized schema.
%
However, these particular operations also happen to be idempotent for any rows
affected, and thus could be reapplied via \eg a cron job regularly to update
revealed data.
%

%
For the large majority of database management operations, users can still reveal
their data disguised prior to these migrations, as if the migration had occurred
with their data present in the database.

%
\begin{comment}
    *row operation
    normalize URL
    set column value to NULL
    fix note formatting

    *table operation
    1 remove 
    add index and column
    add column
    add index
    add story_texts from stories
    create story_texts 
    add index
    add column (and modify it)
    create table
    set value and remove column
    rename column
    add column
    create categories and update columns
    add index
    add index

    *unsupported
    update comments based on count of votes from votes
    memoize score not upvotes

\end{comment}
%%%%%%%%%%%%%%%%
\subsection{Performance of Reveals with Updates}
\label{s:eval:updates}

We next look at how bulk operations and updates affect application performance,
and how updates affect \sys's reveal performance.  Table~\ref{tab:updates} shows
the results.

\paragraph{Bulk Operation Performance.}
First, we look at the cost of performing bulk operations, which applications
incur without \sys.
%
Applying these changes themselves equates to modifying all 120k stories (\eg URL 
normalization) and/or changing the database schema.  The URL normalization
application update takes 5.63s, the longest of all updates.
%
This matches the optimized bulk update time of 6s in the deployed Lobsters
application, which batch updates all stories with the normalized URL. The
unoptimized version that updates stories one-by-one takes 1789s.
%

Adding the \texttt{show\_email} attribute to the \texttt{users} table takes
82ms, and 
%
creating and populating the \texttt{story\_texts} table, and removing a column
from the \texttt{stories} table takes 4.93s.
%

%
\paragraph{Record Update.}
%
Invoking \sys's hook to insert a new application update in \sys's replay log
takes on average 0.58ms, a negligible amount compared to the time to perform the
update/migration. This includes the time to persist the update (one database
insert query) and add it to \sys's in-memory update replay log.
%

\paragraph{Reveal with Updates.}
%
We next evaluate the latency of reveal operations in Lobsters when updates
corresponding to the implemented three bulk operations have been applied after
reveal.
%

%
Applying the three updates to one disguised story row takes 0.2--0.3ms, and
applying the three updates to one disguised user takes $<0.1$ms.
%
The updates do not affect other row types.
%
The URL normalization update requires 0.22ms, where initializing a URL
normalizer object takes $\approx$0.2ms.
%
However, because \sys applies an update to all affected rows of the same table
simultaneously, this cost is amortized by applying updates to all disguised
stories at once. \todo{impl}
%

After applying the update, \sys also must restore the updated rows to
the database. Updates do not affect the cost of restoring rows for all tables
other than stories.  However, restoring stories also requires an additional
query to reveal a \texttt{story\_texts} table row, incurring a 0.1--0.3ms cost
per story.

%
As expected, the effects of performing updates increases with the amount of disguised data a
user wants to reveal.
%
Users with very few (\eg $<5$ posted stories) see little increase in reveal
latency with updates, but revealing users with lots of data shows moderate effects
from updates (\eg revealing a user with 700 stories, 2000 comments, and
\todo{xxx} messages takes on
average 6.9s instead of 5.6s).

%
\section{Summary}
In our evaluation, we used \sys to add seven \xxing transformations to three web
applications. We found that the effort required was reasonable, that \sys's
\xxing and revealing operations are fast enough to be practical, and that they
impose little overhead on normal application operation.
%
