\sys is a first step towards a world in which web services
routinely manage, store, and reveal \xxed user data.
%
In this setting, new questions and directions for research arise.
%

\section{Retention of Disguised Data.}
%
When a user reveals their data, \sys removes it from the disguise table.
%
However, \sys currently retains disguised data until a user reveals it, which
could be forever if users choose to never reveal data.
%
\sys could allow applications to put reasonable,
coarse-grained time limits (\eg 10 years) on disguised data to eventually clean
it up, without leaking fine-grained information about which data was disguised
at the same time.
%Given the amount of data that applications are willing to
%store and retain nearly indefinitely, we imagine that long timeouts will not be
%an issue.


\section{Pseudoprincipal references.}
%
\sys currently supports a global specification for checking and fixing references to
pseudoprincipals.
%
\sys could also support a menu of options, such
as per-table checks and fixes (where the developer to specifies per-table
policies) or per-inserted-object ones (where the developer makes application
modifications to log all added references to pseudoprincipals).

\todo{More discussion about updates? Potential integration of Edna with industry
systems?}
